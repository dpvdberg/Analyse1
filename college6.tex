% !TeX spellcheck = en_US
\documentclass[week=6]{homework}
\usepackage{scrextend}
\date{\today}


\begin{document}
    \maketitle
    \thispagestyle{empty}
    \newpage
    \begin{questions}
		\let\firstquestion\question
		\renewcommand*{\question}{\vspace{7mm}\firstquestion}
        \firstquestion
        Test the given series for convergence or divergence.
        \begin{enumerate}[label=(\alph*)]
        	\item $\displaystyle \sum_{k=2}^{\infty} \frac{1}{\ln k}$
        	
        	% INSERT 1A
        	
        	\item $\displaystyle \sum_{k=1}^{\infty} \frac{1}{\sqrt k + 1.1}$
        	
        	% INSERT 1B
        	
        	\addtocounter{enumi}{2}
        	\item $\displaystyle \sum_{k=1}^{\infty} \frac{1}{k!}$
        	
        	% INSERT 1E
        	
        	\item $\displaystyle \sum_{k=1}^{\infty} \frac{k+1}{k^2}$
        	
        	% INSERT 1F
        	
        	\addtocounter{enumi}{3}
        	\item $\displaystyle \sum_{k=1}^{\infty} \frac{\sin k}{k^2}$
        	
        	% INSERT 1J
        	
        	\item $\displaystyle \sum_{k=1}^{\infty} \frac{\ln k}{k}$
        	
        	% INSERT 1K
        	
        	\addtocounter{enumi}{1}
        	\item $\displaystyle \sum_{k=1}^{\infty} \frac{1}{k^k}$
        	
        	% INSERT 1M
        \end{enumerate}
	    
	    \question
	    Let $\sum a_n$ be a series consisting of positive terms. Prove the following statements:
	    \begin{parts}
	    	\part
	    	\begin{inlinetoprove}
	    		If $\frac{a_{n+1}}{a_n} \xrightarrow{n\to\infty} r < 1$, then $\sum a_n$ is convergent.
	    	\end{inlinetoprove}
	    	\begin{proof}
	    		We first proof the following \textit{Generalized Ratio Test for convergence}
	    		\begin{addmargin}[2em]{0em}
	    			\begin{inlinetoprove}
	    				$\displaystyle \exists_{q\in [0,1)}\exists_{n_0 \in \naturals}\forall_{n\geq n_0}: a_n > 0 \wedge \frac{a_{n+1}}{a_n} \leq q \implies \sum a_k = S \in \reals\,.$
	    			\end{inlinetoprove}
	    			\begin{proof}
	    				The antecedent holds for all $n\geq n_0$, which means that for $n_0$ we have $a_{n_0} > 0$ and
	    				\begin{align*}
		    				\frac{a_{n_0+1}}{a_{n_0}} &\le q \\
		    				\therefore a_{n_0+1} &\le q \cdot a_{n_0}\,.
	    				\end{align*}
	    				and for $n_0 + 1$, using similar reasoning
	    				\begin{align*}
	    					\frac{a_{n_0+2}}{a_{n_0+1}} &\le q \\
	    					\therefore a_{n_0+2} &\le q \cdot a_{n_0+1} \le q^2 a_{n_0}\,,
	    				\end{align*}
	    				Continuing this argument, we find
	    				\begin{align*}
	    					a_{n_0+1} &\le q \cdot a_{n_0} \\
	    					a_{n_0+2} &\le q \cdot a_{n_0+1} \le q^2 a_{n_0} \\
	    					\vdots \quad & \qquad \quad \vdots \\
	    					a_{n_0+k} &\le q^k a_{n_0} \enskip \forall_{k \geq 1}\,.
	    				\end{align*}
	    				But then $\sum_{k=n_0}^{\infty} a_k$ is dominated by the converging series $\sum_{k=0}^{\infty}q^k a_{n_0}$. The latter series converges as it is a geometric series times some constant $a_{n_0}$ and the behavior of the terms preceding $n_0$ is only a finite sum. Using the comparison test we conclude $\sum a_k$ is convergent.
	    			\end{proof}
		    	\end{addmargin}
	    		As $\frac{a_{n+1}}{a_n} \xrightarrow{n\to\infty} r$, we have
	    		\[
		    		\forall_{\epsilon > 0}\exists_{n_0 \in \naturals}\forall_{n \geq n_0}: \left| \frac{a_{n+1}}{a_n} - r \right| < \epsilon\,.
	    		\]
	    		
	    		We have $r < 1$, choosing $\epsilon = 1-r > 0$, we have for $n \geq n_0$:
	    		\begin{alignat*}{3}
	    			 & && \hspace{13pt} \left| \frac{a_{n+1}}{a_n} - r \right| &&< \epsilon\\
	    			 &\therefore -(1-r) &&< \enskip\frac{a_{n+1}}{a_n} - r \enskip &&< 1-r \\
	    			 &\therefore -1+2r &&< \enskip\enskip\enskip \frac{a_{n+1}}{a_n} \enskip &&< 1 \\
	    		\end{alignat*}
	    		Now since $\frac{a_{n+1}}{a_n} < 1$ for $n \geq n_0$ and $a_n > 0$ for all $n$, using the Generalized Ratio Test for convergence we can conclude $\sum a_n$ is convergent. 
	    	\end{proof}
	    	
	    	\part
	    	\begin{inlinetoprove}
	    		If $\frac{a_{n+1}}{a_n} \xrightarrow{n\to\infty} r > 1$, then $\sum a_n$ is divergent.
	    	\end{inlinetoprove}
	    	\begin{proof}
	    		We first proof the following \textit{Generalized Ratio Test for divergence}
	    		\begin{addmargin}[2em]{0em}
	    			\begin{inlinetoprove}
	    				$\displaystyle \exists_{n_0 \in \naturals}\forall_{n\geq n_0}: a_n > 0 \wedge \frac{a_{n+1}}{a_n} \geq 1 \implies \sum a_k \text{ diverges.}$
	    			\end{inlinetoprove}
	    			\begin{proof}
	    				For $n \geq n_0$ we have $a_n > 0$ and
	    				\begin{align*}
	    					\frac{a_{n+1}}{a_n} &\geq 1 \\
	    					\therefore a_{n+1} &\geq a_n\,.
	    				\end{align*}
	    				As $a_{n_0} > 0$, we know $a_n \geq a_{n_0} > 0$ and $\lim_{n\to\infty} a_n \neq 0$. But then $\sum a_n$ can not converge by the divergence test.
	    			\end{proof}
	    		\end{addmargin}
	    		As $\frac{a_{n+1}}{a_n} \xrightarrow{n\to\infty} r$, we have
	    		\[
	    		\forall_{\epsilon > 0}\exists_{n_0 \in \naturals}\forall_{n \geq n_0}: \left| \frac{a_{n+1}}{a_n} - r \right| < \epsilon\,.
	    		\]
	    		
	    		We have $r > 1$, choosing $\epsilon = r-1 > 0$, we have for $n \geq n_0$:
	    		\begin{alignat*}{3}
	    		& && \hspace{13pt} \left| \frac{a_{n+1}}{a_n} - r \right| &&< \epsilon\\
	    		&\therefore -(r-1) &&< \enskip\frac{a_{n+1}}{a_n} - r \enskip &&< r-1 \\
	    		&\therefore \qquad 1 &&< \enskip\enskip\enskip \frac{a_{n+1}}{a_n} \enskip &&< 2r-1 \\
	    		\end{alignat*}
	    		Now since $\frac{a_{n+1}}{a_n} > 1$ for $n \geq n_0$ and $a_n > 0$ for all $n$, using the Generalized Ratio Test for divergence we can conclude $\sum a_n$ is divergent. 
	    	\end{proof}
	    \end{parts}
    
	    \question
	    Let $\sum a_n$ be a series consisting of non-negative terms. Prove the following statements:
	    \begin{parts}
	    	\part
	    	\begin{inlinetoprove}
	    		If $\sqrt[n]{a_n} \xrightarrow{n\to\infty} r < 1$, then $\sum a_n$ is convergent.
	    	\end{inlinetoprove}
	    	\begin{proof}
	    		%INSERT 3A
	    	\end{proof}
	    	
	    	\part
	    	\begin{inlinetoprove}
	    		If $\sqrt[n]{a_n} \xrightarrow{n\to\infty} r > 1$, then $\sum a_n$ is divergent.
	    	\end{inlinetoprove}
	    	\begin{proof}
	    		%INSERT 3B
	    	\end{proof}
	    \end{parts}
    
	    \question
	    Test the given series for convergence or divergence.
	    \begin{enumerate}[label=(\alph*)]
	    	\item $\displaystyle \sum_{k=2}^{\infty} \frac{k^2 2^{k+1}}{3^k}$
	    	
	    	% INSERT 4A
	    	
	    	\item $\displaystyle \sum_{k=1}^{\infty} \frac{(k!)^2}{(2k)!}$
	    	
	    	% INSERT 4B
	    	
	    	\addtocounter{enumi}{2}
	    	\item $\displaystyle \sum_{k=1}^{\infty} \frac{3^{2k+1}}{k^{2k}}$
	    	
	    	% INSERT 4E
	    	
	    	\addtocounter{enumi}{1}
	    	\item $\displaystyle \sum_{k=1}^{\infty} \frac{k!}{k^k}$
	    	
	    	% INSERT 4G
	    	
	    	\addtocounter{enumi}{1}
	    	\item $\displaystyle \sum_{k=1}^{\infty} \left(1 + \frac{1}{k}\right)^k$
	    	
	    	% INSERT 1I
	    \end{enumerate}
     \end{questions}
\end{document}