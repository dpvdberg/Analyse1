% !TeX spellcheck = en_US
\documentclass[week=6]{homework}

\date{\today}


\begin{document}
    \maketitle
    \thispagestyle{empty}
    \newpage
    \begin{questions}
		\let\firstquestion\question
		\renewcommand*{\question}{\vspace{7mm}\firstquestion}
        \firstquestion
        Test the given series for convergence or divergence.
        \begin{enumerate}[label=(\alph*)]
        	\item $\displaystyle \sum_{k=2}^{\infty} \frac{1}{\ln k}$
        	
        	% INSERT 1A
        	
        	\item $\displaystyle \sum_{k=1}^{\infty} \frac{1}{\sqrt k + 1.1}$
        	
        	% INSERT 1B
        	
        	\addtocounter{enumi}{2}
        	\item $\displaystyle \sum_{k=1}^{\infty} \frac{1}{k!}$
        	
        	% INSERT 1E
        	
        	\item $\displaystyle \sum_{k=1}^{\infty} \frac{k+1}{k^2}$
        	
        	% INSERT 1F
        	
        	\addtocounter{enumi}{3}
        	\item $\displaystyle \sum_{k=1}^{\infty} \frac{\sin k}{k^2}$
        	
        	% INSERT 1J
        	
        	\item $\displaystyle \sum_{k=1}^{\infty} \frac{\ln k}{k}$
        	
        	% INSERT 1K
        	
        	\addtocounter{enumi}{1}
        	\item $\displaystyle \sum_{k=1}^{\infty} \frac{1}{k^k}$
        	
        	% INSERT 1M
        \end{enumerate}
	    
	    \question
	    Let $\sum a_n$ be a series consisting of positive terms. Prove the following statements:
	    \begin{parts}
	    	\part
	    	\begin{inlinetoprove}
	    		If $\frac{a_{n+1}}{a_n} \xrightarrow{n\to\infty} r < 1$, then $\sum a_n$ is convergent.
	    	\end{inlinetoprove}
	    	\begin{proof}
	    		%INSERT 2A
	    	\end{proof}
	    	
	    	\part
	    	\begin{inlinetoprove}
	    		If $\frac{a_{n+1}}{a_n} \xrightarrow{n\to\infty} r > 1$, then $\sum a_n$ is divergent.
	    	\end{inlinetoprove}
	    	\begin{proof}
	    		%INSERT 2B
	    	\end{proof}
	    \end{parts}
    
	    \question
	    Let $\sum a_n$ be a series consisting of non-negative terms. Prove the following statements:
	    \begin{parts}
	    	\part
	    	\begin{inlinetoprove}
	    		If $\sqrt[n]{a_n} \xrightarrow{n\to\infty} r < 1$, then $\sum a_n$ is convergent.
	    	\end{inlinetoprove}
	    	\begin{proof}
	    		We know:
	    		\[
		    		\forall_{\epsilon > 0} \exists_{n_0} \forall_{n \ge n_0}: |\sqrt[n]{a_n} - r| < \epsilon 
	    		\]
	    		But then it holds $\forall_{n \ge n_0}$ that:
	    		\begin{align*}
		    		&-\epsilon < \sqrt[n]{a_n} - r < \epsilon \\
		    		& -\epsilon +r < \sqrt[n]{a_n} < \epsilon + r
	    		\end{align*}
	    		We know $r < 1$, so we can choose $\epsilon > 0$ such that $\epsilon + r < 1$. Now define $q = \epsilon + r$, for this chosen $\epsilon$. Then:
	    		\[
					\exists_{q \in [0,1)} \exists_{n_0} \forall_{n \ge n_0} : \sqrt[n]{a_n} < q	    		
	    		\]
	    		But then, by the general root test, $\sum a_n$ converges. 
	    		
	    		Now, by the general root test, $\sum a_n$ diverges. 
	    	\end{proof}
	    	
	    	\part
	    	\begin{inlinetoprove}
	    		If $\sqrt[n]{a_n} \xrightarrow{n\to\infty} r > 1$, then $\sum a_n$ is divergent.
	    	\end{inlinetoprove}
	    	\begin{proof}
	    		We know:
	    		\[
		    		\forall_{\epsilon > 0} \exists_{n_0} \forall_{n \ge n_0}: |\sqrt[n]{a_n} - r| < \epsilon 
	    		\]
	    		But then it holds $\forall_{n \ge n_0}$ that:
	    		\begin{align*}
		    		&-\epsilon < \sqrt[n]{a_n} - r < \epsilon \\
		    		& -\epsilon +r < \sqrt[n]{a_n} < \epsilon + r
	    		\end{align*}
	    		We know that $r > 0$. Now, we can choose $\epsilon > 0$ such that $-\epsilon + r \ge 1$. Now define $q = -\epsilon + r$, for this chosen $\epsilon$. Now:
	    		\[
		    		\exists_{q \in [1,\infty)} \exists_{n_0} \forall_{n \ge n_0} : \sqrt[n]{a_n} \ge q
	    		\]
	    	\end{proof}
	    \end{parts}
    
	    \question
	    Test the given series for convergence or divergence.
	    \begin{enumerate}[label=(\alph*)]
	    	\item $\displaystyle \sum_{k=2}^{\infty} \frac{k^2 2^{k+1}}{3^k}$
	    	
	    	% INSERT 4A
	    	
	    	\item $\displaystyle \sum_{k=1}^{\infty} \frac{(k!)^2}{(2k)!}$
	    	
	    	% INSERT 4B
	    	
	    	\addtocounter{enumi}{2}
	    	\item $\displaystyle \sum_{k=1}^{\infty} \frac{3^{2k+1}}{k^{2k}}$
	    	
	    	% INSERT 4E
	    	
	    	\addtocounter{enumi}{1}
	    	\item $\displaystyle \sum_{k=1}^{\infty} \frac{k!}{k^k}$
	    	
	    	% INSERT 4G
	    	
	    	\addtocounter{enumi}{1}
	    	\item $\displaystyle \sum_{k=1}^{\infty} \left(1 + \frac{1}{k}\right)^k$
	    	
	    	% INSERT 1I
	    \end{enumerate}
     \end{questions}
\end{document}