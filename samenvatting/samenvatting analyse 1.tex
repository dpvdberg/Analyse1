\documentclass[week=1]{homework}

\begin{document}
	\maketitle
	\newpage
	
	\section*{College 1}
	\subsection*{Geordende verzamelingen}
	\Def[Geordende verzamelingen] Een verzameling heet (totaal) geordend als er op $m$ een relatie "$<$" gedefinieerd is ($< \, \subset \, m \times m$) met de eigenschappen:
	\begin{itemize}
		\item voor twee elementen $a,b \in M$ geldt precies één van de beweringen: 
		\begin{itemize}
			\item $a<b$
			\item $b<a$
			\item $a=b$
		\end{itemize}
		\item $\forall a,b,c\in M: a<b \, \wedge \, b<c \Rightarrow a<c$ 
	\end{itemize}
	\Vb $\naturals \quad \complex \quad \rationals \quad \reals$
	
	\Not \begin{itemize}
		\item $a\le b \Leftrightarrow a < b \, \vee \, a=b$ 
		\item $a>b \Leftrightarrow b<a$
		\item $a\ge b \Leftrightarrow a > b \, \vee \, a=b$ 
	\end{itemize}
	
	\Not \ul{intervallen}  
	
	Stel $a,b \in M, \, a<b$
	\begin{itemize}
		\item open $\rightarrow (a,b):= \{x\in M | a < x < b \} $
		\item half-open $\rightarrow [a,b):= \{x\in M | a \le x < b \}$
		\item gesloten $\rightarrow [a,b]:= \{x\in M|a \le x \le b\}$
	\end{itemize}
	
	\Def[Begrensd] Een niet lege deelverzameling $A \in M$ heet "begrensd" naar boven/beneden als er een $g \in M$ is met $a \le g$/$a \ge g \forall a \in A$. G heet dan een bovengrens/ondergrens voor A. 
	
	\Def[Maximum en minimum] Zij $A \subset M$. Een element $m \in A$ heet "maximum/minimum" van A of grootste/kleinste element van A als $m$ ook bovengrens/ondergrens is. 
	
	\Not $\max A$,  $\min A$
	
	\Stel Elke verzameling $A \subset M$ heeft hooguit \'e\'en maximum/minimum.
	
	\Letop Er hoeft geen maximum van een begrensde verzameling te bestaan! (zie open intervallen)
	
	\Stel De verzameling $A = (0,1)$ heeft geen maximum. 
	
	\vspace{5mm}
	\Def[Supremum en infimum] Zij $A \subset M$ een niet lege, naar boven/beneden begrensde verzameling. De kleinste bovengrens/grootste ondergrens heten supremum/infimum. 
	
	\Stel Als $A \subset M$ een maximum/minimum heeft dan is $\sup A = \max A$/$\inf A = \min A$.
	
	\Stel Zij $A \subset B \subset M$. Stel dat $\sup A$ en $\sup B$ bestaan. Dan $\sup A \subseteq \sup B$.
	
	\subsection*{Geordende lichamen}
	\Stel 12 axiomas (zie boek)
	
	\Def[Dicht liggen] Een verzameling getallen $M$ ligt dicht als in elk interval $(a,b)$ met $a<b$, $a,b \in M$ liggen oneindig veel getallen uit $M$.
	
	\Stel De rationale getallen liggen dicht. 
	
	\Stel $\rationals$ is niet volledig geordend! 
	
	
	\newpage
	\section*{College 2}
	\Stel $\reals$ is een volledig geordend lichaam. 
	
	\Def[Axioma 13] Iedere niet lege naar boven begrensde deelverzameling $S$ van $\reals$ heeft een supremum, en dit $\sup \in \reals$. 
	
	\Stel Elke naar beneden begrensde niet lege verzameling heeft een infimum. 
	
	\Stel[1.7.8] $\naturals$ is niet begrensd naar boven als deelverzameling van $\reals$.  
	
	\Stel $\forall \epsilon > 0$ is er een $n \in \naturals$ zo dat $\frac{1}{n} < \epsilon$. 
	
	\Stel Zij $a,b \in \reals$ met $a < b$. Er is in het interval $(a,b)$ zowel een rationaal getal als een irrationaal getal. 
	
	\Gevolg Er zijn oneindig veel rationele en irrationele getallen. 
	
	\subsection*{Afstand}
	\Def $d(a,b) = |a-b|$. Eigenschappen:
	\begin{itemize}
		\item $\forall a,b \in \reals$ geldt $d(a,b) = d(b,a) \ge 0$. 
		\item $d(a,b) = 0 \Leftrightarrow a = b$.
	\end{itemize}
	
	\Stel[Driehoeksongelijkheid] $|a+b| \le |a| + |b|$
	
	\Stel[Inverse driehoeksongelijkheid] $| |a| - |b| | \le |a-b|$
	
	\Stel $|a-b| \le |a-c| + |c-b|$
	
	\subsection*{Rij}
	\Def[Rij] Zij $M$ een niet lege deelverzameling. Een afbeelding van $\naturals$ of $\naturals_+$ naar $M$ heet een rij in $M$. 
	
	\Not \begin{itemize}
		\item $a_n$
		\item $(a_0, a_1, a_2, ...)$
		\item $(a_n)$
		\item $(a_n)n \in \naturals$
		\item $(a_n)_{n=0}^\infty$
	\end{itemize}
	
	\Def \begin{itemize}
		\item Een rij $(a_k)$ heet naar boven/beneden begrensd als de verzameling ${(a_k)|k \in \naturals}$ naar boven/beneden begrensd is. 
		\item Een rij $(a_k)$ heet stijgend als $a_k \le a_{k+1}$ voor alle $k \in \naturals$. 
		\item Een rij $(a_k)$ heet strikt stijgend als $a_k < a_{k+1}$ voor alle $k \in \naturals$. 
		\item Een rij $(a_k)$ heet dalend als $a_k \ge a_{k+1}$ voor alle $k \in \naturals$. 
		\item Een rij $(a_k)$ heet strikt dalend als $a_k > a_{k+1}$ voor alle $k \in \naturals$.
		\item Een rij is periodiek met periode $p \in \naturals_+$ als $a_{k+p} = a_k$ voor alle $k \in \naturals$.
	\end{itemize}
	
	\Def[Limietdefinitie] $\forall \epsilon > 0 \exists n_0 \in \naturals: \forall n \ge n_0: |x_n - x^*|< \epsilon$
	
	\Not \begin{itemize}
		\item $(a_k)$ convergeert naar $a^*$ 
		\item $(a_k)$ heeft limiet $a^*$
		\item $\lim_{n \rightarrow \infty} a_k = a^*$
		\item $a_k \rightarrow a^*$ als $k \rightarrow \infty$
		\item $a_k \xrightarrow{ k \rightarrow \infty} a^*$ 
	\end{itemize}
	
	\Stel Elke rij heeft hooguit \'e\'en limiet. 
	
	\newpage
	\section*{College 3}
	\subsection*{Eigenschappen van convergente rijen}
	\Stel[2.1.11] Elke convergente rij is begrensd. 
	
	\Stel[2.4.4] Zij $(a_n)$ stijgend en begrensd (naar boven). Dan is $(a_n)$ convergent, en $\lim_{n \rightarrow \infty} \{a_n \mid n \in \naturals \}$.
	
	\Stel[Limietstellingen] Zij $(a_n) (b_n) $ rijen met $a_n \rightarrow a^*, b_n \rightarrow b^*, c \in \reals$. Dan: 
	\begin{itemize}
		\item $a_n + b_n \rightarrow a^* + b^*$
		\item $a_n b_n \rightarrow a^* b^*$ 
		\item Als $a^* \neq 0$, $a_n \neq 0 \forall n$ dan: $\frac{1}{n} \rightarrow \frac{1}{a^*}$
		\item Als $\exists n_0 \in \naturals \forall n \ge n_0: a_n \le b_n$ dan $a^* \le b^*$. 
	\end{itemize}
	
	\Stel[Insluitstelling] Zij $(a_n), (b_n), (c_n)$ rijen met $a_n \le b_n \le c_n$ voor $n$ voldoende groot. Veronderstel $a_n \rightarrow L$, $c_n \rightarrow L$. Dan ook $b_n \rightarrow L$.
	
	\Stel Zij $(a_n), (b_n)$ rijen met $a_n \rightarrow 0$ en $(b_n)$ begrensd. Dan $a_n b_n \rightarrow 0$. 
	
	\subsection*{Oneigenlijke limieten}
	\Def $\lim_{n \rightarrow \infty} a_n = \infty$: 
	\[
		\forall M \in \reals: \exists n_0 = n_0(M): a_n > M \forall n > n_0
	\]
	
	Analoog voor $\lim_{n \rightarrow \infty} a_n = - \infty$
	
	\Stel[2.3.3, 2.3.6] Zij $(a_n), (b_n)$ rijen met $a_n \rightarrow \pm \infty$, $(b_n)$ begrens. Dan:
	\begin{itemize}
		\item $a_n + b_n \rightarrow \pm \infty$ 
		\item $\frac{1}{a_n} \rightarrow 0$
	\end{itemize}
	
	\Stel[Standaardlimieten] Zij $a \in \reals$
	\begin{itemize}
		\item $\lim_{n \rightarrow \infty} a^n \begin{cases}
			0 &\text{als } \mid a \mid < 1\\
			+\infty &\text{als } a > 1
		\end{cases}$
		\item Zij $a \in \reals$, $\mid a \mid > 1$, $k \in \naturals$ vast. Dan $\lim_{n \rightarrow \infty} \frac{n^k}{a^n} = 0$.
		\item $\sqrt[n]{n} \rightarrow 1$
	\end{itemize}
	 
	\newpage
	\section*{College 4}
	\subsection*{Het getal $e$}
		\Def $e_n = (1 + \frac{1}{n})^n$
		
		\Stel De rij $(e_n)$ is convergent. De limiet heet $e$. 
		
		\Stel[Ongelijheid van Bernoulli] Zij $n \in \naturals$, $x > -1$, dan $(1 + x)^n \ge 1 + nx$.	
	
	\subsection*{Deelrijen}
	\Def Zij $(a_n)$ een rij. Zij $(n_0, n_1, n_2, ...)$ een strikt stijgende rij in $\naturals$. De rij $(a_{n_k})$ heet een deelrij van de rij $(a_n)$ (met indexrij $(n_k)$). 
	
	\Letop Als de rij begrensd is, is de deelrij dat ook, maar dat betekend niet dat de deelrijen ook een limiet hebben. 
	
	\Stel[Bolzano-Weierstrass] Elke begrensde rij bevat een convergente deelrij. 
	
	\Def Zij $(a_n)$ een rij. Een getal $y^*$ heet verdichtingspunt als $(a_n)$ een deelrij $(a_{n_l})$ heeft met $a_{n_l} \rightarrow y^*$. 
	
	\Letop Convergente rijen hebben precies één verdichtingspunt. 
	
	\subsection*{Cauchyrijen}
	\Def Een rij $(a_n)$ heet Cauchyrij als: 
	\[
		\forall \epsilon > 0: \exists n_0 = n_0(\epsilon): \forall m,n \ge n_0: \mid a_m - a_n \mid < \epsilon 
	\]
	
	\Letop Er is in de definitie geen sprake van een limiet, maar elke convergente rij is cauchyrij (zie beneden).
	
	\Stel Elke cauchyrij is begrensd. 
	
	\Stel Elke cauchyrij die een convergente deelrij heeft is convergent. 
	
	\Stel $(a_n)$ cauchyrij $\Leftrightarrow (a_n)$ convergent
	
	\Letop Dit geldt alleen voor rijen in $\reals$.
	
	\newpage
	\section*{College 5}
	\subsection*{Reeksen in $\reals$}
	\Def Zij $(a_0, a_1, a_2, ...)$ een rij in $\reals$. \begin{align*}
		s_1 &= a_1 \\
		s_2 &= a_1 + a_2 \\
		&... \\
		s_n &= \sum_{k=1}^{n} a_k \text{partiële sommen}
	\end{align*}
	$(s_n)$ is een rij van partiële sommen. Eigenschappen:
	\begin{itemize}
		\item Als $(s_n)$ convergent naar $s \in \reals$, dan zeggen we dat de reeks $s = \sum_{k=1}^{n} a_k$ convergeert naar waarde $s = \lim_{n \rightarrow \infty} \sum_{k=1}^{n} a_k$. We schrijven dan ook $s = \sum_{k=1}^{\infty} a_k$. 
		\item Als $(s_n)$ divergeert dan heet $\lim_{n \rightarrow \infty} \sum_{k=1}^{n} a_k$ divergent. Als $s_n \xrightarrow{n \rightarrow \infty} \pm \infty$, dan schrijven we $\sum_{k=1}^{\infty} a_k = \pm \infty$.
		\item Een reeks kan ook alterneren. 
	\end{itemize}
	
	\Letop Rekenen met divergente reeksen kan tot tegenspraken leiden! Niet doen dus!
	
	\Stel[7.1.9] Als $\sum_{k=1}^{\infty} a_k$ convergeert dan $a_k \xrightarrow{k \rightarrow \infty} 0$. 
	
	\Letop Uit $a_k \rightarrow 0$ volgt niet $\sum_{k=1}^{\infty} a_k$ convergent!
	
	\Stel Zij $\sum_{k=1}^{\infty} a_k$ convergent, en $k_0 \in \naturals_+$. Dan ook $\sum_{k=k_0}^{\infty} a_k$ convergent en 	
	\[
		\sum_{k=k_0}^{\infty} a_k = \sum_{k=1}^{\infty} a_k - \sum_{k=1}^{k_0 - 1} a_k
	\] 
	
	\Stel Zij $\sum_{k=1}^{\infty} a_k$ convergent. Dan $\lim_{n \rightarrow \infty} \sum_{k=n}^{\infty} a_k = 0$.
	
	\subsection*{Convergentie criteria voor reeksen}
	\Def De reeks $\sum_{k=1}^{\infty} a_k$ heet absoluut convergent als $\sum_{k=1}^{\infty} \mid a_k \mid $ convergent. 
	
	\Stel Elke absoluut convergente reeks is convergent (in $\reals $). 
	
	\subsection*{Reeksen met niet negatieve termen}
	\Stel Zij $\sum_{k=1}^{\infty} a_k$ een reeks met $\forall k: a_k \ge 0 $, $s_n = \sum_{k=1}^{n} a_k$. Dan $(s_n)$ stijgend. Ook zijn de volgende beweringen equivalent:
	\begin{itemize}
		\item $(s_n)$ convergent
		\item $(s_n)$ begrensd naar boven
		\item $(s_n)$ heeft een naar boven begrensde deelrij
	\end{itemize}
	\Stel[Majorantencriterium] Zij $\forall k \in \naturals_+: 0 \le a_k \le b_k$. Stel $\sum_{k=1}^{\infty} b_k$ convergent. Dan $\sum_{k=1}^{\infty} a_k$ convergent, en $\sum_{k=1}^{\infty} a_k \le \sum_{k=1}^{\infty} b_k$. 
	
	\Def $\sum_{k=1}^{\infty} b_k$ heet de convergente majorante voor $\sum_{k=1}^{\infty} a_k$. 
	
	\Stel[Minorantencriterium] Zij $\forall k \in \naturals_+: 0 \le a_k \le b_k$. Stel $\sum_{k=1}^{\infty} a_k$ divergent, dan $\sum_{k=1}^{\infty} b_k$ divergent. 
	
	\Def $\sum_{k=1}^{\infty} a_k$ heet een divergente minorante voor $\sum_{k=1}^{\infty} b_k$.
	
	\subsection*{Standaardreeksen}
	\Def[Meetkundige reeksen] 
	\[
		\sum_{k=0}^{\infty} q^k = 1 + q + q^2 + ... = \begin{cases}
			\frac{1}{1-q} &(\mid q \mid < 1) \\
			\text{divergent} &(\mid q \mid ge 1)
		\end{cases}
	\]
	\Def[Hyperharmonische reeksen] Zij $\alpha > 0$ vast. De reeks $\sum_{k=1}^{\infty} \frac{1}{k^{\alpha}}$ heet hyperharmonische reeks. Zij is convergent voor $\alpha > 1$ en divergent voor $\alpha \le 1$. 
	
	\newpage
	\section*{College 6}
	
	\subsection*{Convergentiecriteria}
	\Stel[Quoti\"entcriterium] 
	\begin{itemize}
		\item $\exists q \in [0,1): \exists n_0 \in \naturals: \forall n \ge n_0: \frac{a_{n+1}}{a_n} \le q$. 
		\item Als $\exists n_0 \in \naturals: \forall n \ge \frac{a_{n+1}}{a_n} \ge 1$, dan $\sum_{k=1}^{\infty} a_k$ divergent. 
	\end{itemize}
	
	\Gevolg \begin{itemize}
		\item Als $\lim_{n \rightarrow \infty \frac{a_{n+1}}{a_n}} = r$, $r < 1$ dan $\sum_{k=1}^{\infty} a_k$ convergent. 
		\item Als $\lim_{n \rightarrow \infty \frac{a_{n+1}}{a_n}} = r$, $r > 1$ dan $\sum_{k=1}^{\infty} a_k$ divergent. 
	\end{itemize}
	
	\Letop $\frac{a_{n+1}}{a_n} < 1$ niet voldoende voor convergentie! Het limiet echter wel. 
	
	\Stel[Wortelcriterium] \begin{itemize}
		\item Als $\exists q \in [0,1): \exists n_0 \in \naturals: \forall n \ge n_0: \sqrt[n]{a_n} \le q$ 
		\item Als $\exists n_0 \in \naturals \forall n \ge n_0: \sqrt[n]{a_n} \ge 1$ dan $\sum_{k=1}^{\infty} a_k$ divergent. 
	\end{itemize}
	
	\Gevolg \begin{itemize}
		\item Als $\sqrt[n]{a_n} \xrightarrow{n \rightarrow \infty} r$, $r < 1$ dan $\sum_{k=1}^{\infty} a_k$ convergent. 
		\item Als $\sqrt[n]{a_n} \xrightarrow{n \rightarrow \infty} r$, $r > 1$ dan $\sum_{k=1}^{\infty} a_k$
	\end{itemize}
	
	\subsection*{Alternerende reeksen}
	\Def Een reeks $\sum_{k=1}^{\infty} b_k$ heet alternerend als $b_k = (-1)^k a_n$ of $b_k = (-1)^{k+1} a_n$ met $a_n \ge 0$. Zonder verlies van algemeenheid: we beschouwen $\sum_{k=0}^{\infty} (-1)^k a_k = a_0 - a_1 + a_2 - a_3 + ...$
	
	\Stel[Criterium van Leibniz] Neem aan: \begin{itemize}
		\item $a_{k+1} \le a_k$ 
		\item $\lim_{n \rightarrow \infty} a_k = 0$
		\item $\sum_{k=1}^{\infty} (-1)^k a_k$ alternerend $(a_ \ge 0)$.
	\end{itemize}
	Dan $s =  \sum_{k=1}^{\infty} (-1)^k a_k$ convergent en 
	\[
		\mid s - \sum_{k=0}^{l} (-1)^k a_k \mid \le a_{l+1}
	\]
	
	\Letop $\sum_{k=1}^{\infty} (-1)^k a_k = 1 - \frac{1}{2} + \frac{1}{2} - \frac{1}{3} + \frac{1}{4} - \frac{1}{4} + \frac{1}{8} - \frac{1}{5} + ...$ Hierbij kan je Leibniz niet toepassen!
	
	\subsection*{Herordening van reeksen:}
	\Def $\pi: \naturals \rightarrow \naturals$ permutatie. De reeks $\sum_{k=0}^{\infty} a_{\pi(k)}$ heet een herordening van de reeks $\sum_{k=0}^{\infty} a_k$. 
	
	\Letop Voor een voorwaardelijke convergente reeks zijn er herordeningen die: 
	\begin{itemize}
		\item niet convergeren 
		\item convergeren naar een willekeurig voorgegeven waarde
		\item divergeren naar $+ \infty$ en naar $- \infty$
	\end{itemize}
	
	\Def Voor een voorwaardelijk convergente reeks geldt: $\sum_{k=1}^{\infty} a_k^+ = \infty$, $\sum_{k=1}^{\infty} a_k^- = - \infty$
	\begin{align*}
		a^+ = \begin{cases}
			a &a>0 \\
			0 &a\le 0
		\end{cases} \\
		a^- = \begin{cases}
			0 &a>0 \\
			a &a\le0
		\end{cases}
	\end{align*}
	
	\Stel[7.4.15] $f: \naturals \rightarrow \naturals$ bijectief, $\sum_{k=1}^{\infty} a_{f()}$ herordening. Zij $\sum_{k=0}^{\infty} a_k$ absoluut convergent. Dan convergeert ook $\sum_{k=0}^{\infty} a_{f(k)}$ en $\sum_{k=0}^{\infty} a_{f(k)} = \sum_{k=0}^{\infty} a_k$. 
	
	\newpage
	\section*{College 7}
	
	\subsection*{Producten van absoluut convergente reeksen}
	\Def[Cauchieproduct] zie pp
	
	\Stel Veronderstel $\sum_{k=0}^{\infty} a_k = A$, $\sum_{l=0}^{\infty} b_l = B$ absoluut convergente reeksen. Dan $c = \sum_{n=1}^{\infty} (\sum_{k+l=n} a_k b_l)$ absoluut convergent. Ofwel $c_n = \sum_{k=0}^{n} a_k b_{n-k}$ en $c_n$ convergeert met waarde $A \cdot B$.
	
	\subsection*{De exponentiaalfunctie}
	\Stel Voor $x \in \reals$ is de reeks $\sum_{k=0}^{\infty} \frac{x^k}{k!}$ absoluut convergent. 
	
	\Def Voor $x \in \reals$ zij $\exp(x) = \sum_{k=0}^{\infty} \frac{x^k}{k!}$. Eigenschappen:
	\begin{itemize}
		\item $\exp: \reals \rightarrow (0, \infty)$ strikt stijgend, continu
		\item Inverse: $\ln(0,\infty) \rightarrow \reals $
	\end{itemize}
	
	\Stel \begin{itemize}
		\item $\exp(1) = e$ 
		\item $\exp(x+y) = \exp(x) \cdot \exp(y)$
	\end{itemize}
	
	\Gevolg 
	\[
		\forall a \in \rationals: \exp(ax) = \exp(x)^a
	\]
	
	\Letop Algemene machten: $x > 0, y \in \reals, x^y = e^{y \cdot \ln(x)}$
	
	\newpage
	\section*{College 8}
	
	\subsection*{Herhaling en verdieping limieten van functies}
	\Def Zij $D \subset \reals$, $a \in \reals$ heet verdichtingspunt van $D$ als: $\forall \delta > 0 \exists x \in D: 0 < \mid x-a \mid < \delta$. 
	
	\Not $a \in D'$ verzameling van alle verdichtingspunten
	
	\Stel Zij $D \subset \reals$, $a \in \reals$. Dan:
	\[
		a \in \reals' \Leftrightarrow \exists \text{rij } (x_n) \text{in } D \backslash \{a\} \text{met } x_n \rightarrow a
	\]
	
	\Def Zij $f: D \rightarrow \reals$, $a \in D'$. 
	\[
		\lim_{x \rightarrow a} f(x) = L \Leftrightarrow \forall \epsilon > 0: \exists \delta: \forall x \in D: 0 < \mid x - a \mid < \delta \Rightarrow \mid f(x) - L \mid < \epsilon
	\]
	
	\Stel 
	\[
		\lim_{x \rightarrow a} f(x) = L \Leftrightarrow \forall \text{ rijen } (x_n) \text{ in } D \backslash \{ a \} \text{ met } x_n \rightarrow a \text{ geldt } f(x_n) \rightarrow L. 
	\]
	
	\Stel[verdere limieten] 
	\[
		\lim_{x \rightarrow \infty} f(x) = L \Leftrightarrow \forall \delta > 0: \exists x_0 \in \reals: x \ge x_0 \Rightarrow \mid f(x) - L \mid < \epsilon 
	\]
	Analoog voor $x \rightarrow -\infty$
	
	\Stel[Eenzijdige limieten]
	\[
		\lim_{x \downarrow a} f(x) = L \Leftrightarrow \lim_{x \rightarrow a} (f \mid_{D \cap [a,\infty]}) = L
	\]
	
	\Stel[Oneigenlijke limieten]
	\[
		\lim_{x \rightarrow a} f(x) = \pm \infty \Leftrightarrow 
	\]
	
	\subsection*{Limietstellingen voor functies}
	\Vb $g,f: D \rightarrow \reals$, $a \in (D \cap [a,\infty))'$
	\[
		\lim_{x \downarrow a} f(x) = L > 0, \lim_{x \downarrow a} g(x) = - \infty \Rightarrow \lim_{x \downarrow a f(x) \cdot g(x) = - \infty}
	\]
	
	\subsection*{Continue functies}
	\Def Zij $f: D \rightarrow \reals$, $a \in D' \cap D$. f heet continu bij a d.e.s.d.a:
	\begin{itemize}
		\item $\lim_{x \rightarrow a} f(x) = f(a) \Leftrightarrow $
		\item $\forall \epsilon > 0: \exists \delta > 0: \forall x \in D: \mid x-a \mid < \delta \Rightarrow \mid f(x) - f(a) < \epsilon$
	\end{itemize}
	
	\Stel f continu bij a $\Leftrightarrow \forall \text{ rijen } (x_n) in D: x_n \rightarrow a \Rightarrow f(x_n) \rightarrow f(a) $ ofwel $\lim_{n \rightarrow \infty} f(x_n) = f( \lim_{n \rightarrow \infty} x_n)$
	
	\Def Zij $D \subset \reals$ met $D \subset D'$, $f: D \rightarrow \reals$. f heet continu op $D$ als f continu is in alle punten van $D$. 
	
	\subsection*{Eigenschappen van continue functies}
	\Stel[Tussenwaardestelling] Zij $a,b \in \reals$, $a<b$, $f: [a,b] \rightarrow \reals$ continu. Zij $f(a) < y < f(b)$. Dan $\exists c \in (a,b): f(c) = y$. 
	
	\Stel[Maximaaleigenschap] Zij $a,b \in \reals$, $f:[a,b] \rightarrow \reals$ continue. Dan $\exists x^* \in [a,b]: f(x^*) = \sup \{f(x) \mid x \in [a,b] \} = \max \{f(x) \mid x \in [a,b] \}$
	
	\Stel[Minimaaleigenschap] Analoog
	
	\newpage
	\section*{College 9}
		
	\subsection*{Differentieerbare functies}
	
	\Def[Differentieerbaarheid] $f : D \to \reals$, $a \in D \cap D'$ met $D \subset \reals$.
	$f$ heet differentieerbaar bij $a$ als 
	\[
		\lim_{x \to a} \frac{f(x) - f(a)}{x-a} = \lim_{h \to 0} \frac{f(a+h) - f(a)}{h}
	\]
	bestaat. Deze heet dan afgeleide van $f$ in $a$. 
	
	\Not $f'(a)$, $\frac{df}{dx}(a)$
	
	\Stel[Eigenschappen/rekenregels] 
	\begin{itemize}
		\item $f$ d'baar bij a $\Rightarrow$ f continu bij a
		\item $f,g$ d'baar bij a $\Rightarrow$ $f + g$, $f \cdot g$ d'baar bij a, en $(f+g)'(a) = f'(a) + g'(a)$ en $(f \cdot g)'(a) = f'(a) g(a) + g'(a)f(a)$. 
	\end{itemize}
	
	\Stel[Kettingregel] $I, J$ intervallen., $f : I \to \reals$, $R(f) \subset J$, $g : J \to \reals$. $f$ d'baar bij $a \in I$, $g$ d'baar bij $f(a)$. Dan $g \circ f : I \to \reals$ d'baar bij $a$, en 
	\[
		(g \circ f)'(a) = g'(f(a)) \cdot f'(a) = ((g' \circ f) f') (a)
	\]
	
	\Stel Zij $f: (a,b) \to \reals$ d'baar, $c \in (a,b)$ met $f(c) = \min f(x)$ met $x \in (a,b)$. Dan $f'(c) = 0.$ Geldt ook voor $\max$.
	
	\Stel[Rolle] $f:[a,b] \to \reals$ continu, $f$ op $(a,b)$ d'baar, $f(a) = f(b) = z.$ Dan 
	\[
		\exists_{c \in (a,b)} : f'(c) = 0.
	\]
	
	\Stel[Middelwaardestelling] $f:[a,b] \to \reals$ continu, $f$ op $(a,b)$ d'baar. Dan:
	\[
		\exists_{c \in (a,b)} : f'(c) = \frac{f(b) - f(a)}{b-a}
	\]
	\subsection*{Monotoniegedrag}
	
	\Stel $f : [a,b] \to \reals$ continu, d'baar op $(a,b)$. 
	\begin{enumerate}
		\item $f$ stijgend $\Rightarrow f'(x) \ge 0 \quad \forall_{x \in (a,b)}$
		\item $f'(x) \ge 0 \quad \forall_{x \in (0,1)} \Rightarrow f$ stijgend 
	\end{enumerate}
	Analoog dalend.
	
	\Stel[Stelling van Taylor] $I$ interval, $f$ is $n+1$ keer d'baar op $I$, $x, a \in I$. $\exists_{\epsilon \in (a,x)} \text{ of } (x,a)$. 
	\[
		f(x) = \sum_{k=0}^{n} \frac{f^k(a)}{k!} (x-a)^k + \frac{f^{(n+1)}(\epsilon)}{(n+1)!} (x-a)^{n+1}
	\]
	Ook $\exists_{\theta \in (0,1)}: R_n = \frac{f^{(n+1)} (a + \theta(x-a))}{(n+1)!}(x-a)^{n+1}$. 	
	
	\subsection*{Taylorreeksen}
	
	\Def Zij $I$ open interval in $\reals$, $f : I \to \reals$ willekeurig vaak d'baar. $a,x \in I$. 
	Er geld dat $\forall n \in \naturals : f(x) = \sum_{k=0}^{n} \frac{f^k(a)}{k!} (x-a)^k + R_n(x)$ en $n \to \infty$
	\[
		f(x) = \sum_{k=0}^{\infty} \frac{f^k(a)}{k!} (x-a)^k \Leftrightarrow R_n(x) \to 0
	\]
	De taylorreeks voor $f$ rond $a$ is 
	\[
	f(x) = \sum_{k=0}^{\infty} \frac{f^k(a)}{k!} (x-a)^k
	\]
	
	\newpage
	\section*{College 10}

	\subsection*{Functierijen}

	\Def Zij $D \subset \reals$. Een afbeelding $f : \naturals x D \to \reals$ heet een functierij op $D$. 
	
	\Not $f(n,x)$, $f_n(x)$, $f_n$ op $D$
	
	\subsection*{Puntsgewijze convergentie}
	
	\Def Een functierij $(f_n)$ op $D$ heet puntsgewijs convergent als voor alle x in $D$ de rij getallen $f_n(x)$ convergent is. 
	
	\Not $f_n$ is rij functies, maar $(f_n(x))$ is rij getallen
	
	\Def Zij $(f_n)$ op $D$ puntsgewijs convergent. De functie $f^* : D \to \reals$ gegeven door $f^*(x) = \lim_{n \to \infty} f_n(x)$ heet puntsgewijze limiet van de functierij $(f_n)$.
	
	\Not $f_n \xrightarrow{n \to \infty} f^*$ puntsgewijs
	
	\subsection*{Uniforme convergentie}
	
	\Def Een functierij $(f_n)$ op $D$ heet uniform convergent naar een functie $f^* : D \to \reals$ als $f_n - f^*$ begrensd voor n voldoende groot en 
	\[
		\|f_n - f^* \|_{\infty} = \sup_{x \in D} |f_n(x) - f^*(x)| \xrightarrow{n \to \infty} 0
	\]
	
	\Stel Equivalent:
	\begin{itemize}
		\item $\forall_{\epsilon > 0} \exists_{n_0(\epsilon) \in \naturals} \forall_{n \ge n_0} : \|f_n - f^*\|_{\infty} < \epsilon$ 
		\item $\forall_{\epsilon > 0} \exists_{n_0(\epsilon) \in \naturals} \forall_{n \ge n_0} : \sup_{x \in D} |f_n(x) - f^*(x)| < \epsilon$ 
		\item $\forall_{\epsilon > 0} \exists_{n_0(\epsilon) \in \naturals} \forall_{n \ge n_0} \forall_{x \in D} : |f_n(x) - f^*(x)| < \epsilon$ 
	\end{itemize}
	
	\Letop puntsgewijze convergentie:
	$\forall_{x \in D} \forall_{\epsilon > 0} \exists_{n_0(\epsilon, x) \in \naturals} \forall_{n \ge n_0} |f_n(x) - f^*(x)| < \epsilon$ 
	
	\newpage
	\section*{College 11}
	
	\subsection*{Convergentie van functierijen}
	
	\Stel Zij $(f_n)$ functierij op $D$. $f_n \to f^*$ uniform. Als $f_n$ continu voor alle voldoende grote n, dan is ook $f^*$ continu. 
	
	\Letop Uniforme convergentie behoud niet noodzakelijk differentieerbaarheid. 
	
	\Stel Zij $I \subset \reals$ een interval. Zij $(f_n)$ functierij op $I$. Veronderstel:
	\begin{itemize}
		\item $f_n$ continu diff'baar op $I$ voor alle n. 
		\item $(f_n)$ puntsgewijs convergent op $I$ naar $f: I \to \reals$. 
		\item $(f_n)'$ uniform convergent naar $g : I \to \reals$.
	\end{itemize}
	Dan is $f$ continu d'baar op $I$ en $f' = g$. Oftewel 
	\[
		\left(\lim_{n \to \infty} f_n \right)' = \lim_{n \to \infty} f_n'
	\]
	
	\Stel[Criterium van Cauchy] Zij $D \subset \reals$ en zij $(f_n)$ functierij op $D$ zodanig dat $\forall_{\epsilon > 0} \exists_{n_0} \forall_{n,m \ge n_0} \|f_n - f_m \|_{\infty} < \epsilon $.
	
	\Stel[Criterium van Dini] Zij $[a,b]$ begrensd, gesloten interval, $(f_n)$ functierij op $[a,b]$. Ook $f^* : [a,b] \to \reals$. Veronderstel:
	\begin{enumerate}
		\item $f_n \to f^*$ puntsgewijs
		\item $f_n$ continu voor alle n, $f^*$ continu
		\item $(f_n)$ stijgende rij, d.w.z. $\forall_{x \in [a,b]} \forall_{n,m \in \naturals} : n > m \Rightarrow f_n(x) \ge f_m(x)$
	\end{enumerate}
	Dan $f_n \to f^*$ uniform.
	
	\subsection*{Functiereeksen}

	\Def Zij $(f_k)$ functierij op $D \subset \reals$. Zij $S_n$ op $D$ de functierij van de bijbehorende partiele sommen: $S_n(x) = \sum_{k=0}^{n} f_k(x)$, $x \in D$. Als $S_n$ puntsgewijs/uniform convergent is op $D$, dan heet de bijbehorende $\infty$-reeks $S(x) = \sum_{=0}^{\infty} f_k(x)$ een puntsgewijs/uniform convergente functiereeks op $D$. 

	\newpage
	\section*{College 12}
		
	\subsection*{Eigenschappen van functiereeksen}
	
	\Stel[Continuiteit van functiereeksen] Zij $\sum_{k=1}^{\infty} f_k$ een uniform convergente functiereeks op $D \subset \reals$. Zij $f_k : D \to \reals$ continu $\forall_{k \in \naturals}$. Dan is ook $s: D \to \reals$ gegeven door $s(x) = \sum_{k=1}^{\infty} f_k$ continu. 
	
	\Stel[Differentiatie van functiereeksen] Zij $I \subset \reals$ interval. $\sum_{k=1}^{\infty} f_k$ puntsgewijs convergente functiereeks op $I$. $f_k$ d'baar $\forall_{k \in \naturals}$. $\sum_{k=1}^{\infty} f_k'$ uniform convergent. Dan is ook $s: I \to \reals$ gegeven door $s(x) = \sum_{k=1}^{\infty} f_k$ d'baar, en $s' = \sum_{k=1}^{\infty} f_k'$. 
	
	\Stel[Convergentiecriterium van Weierstrass] Zij $\sum f_k$ functiereeks op $D \subset \reals$. Als $\sum \|f_k\|_\infty$ convergent, dan $\sum f_k$ uniform convergent. 
	
	\Stel[M-criterium van Weierstrass] Stel $|f_k(x)| \le M_k \quad \forall_{k \in \naturals}$, $x \in D$ en $\sum M_k$ convergent. Dan $\sum f_k$ uniform convergent. 
	
	\subsection*{Machtreeksen}
	
	\Def Een functiereeks op $\reals$ van de vorm $\sum_{k=0}^{\infty} a_k(x-x_0)^k$ met $a_k,x \in \reals$ heet machtreeks. 
	
	\Letop Bijzonder belangrijk, want:
	\begin{itemize}
		\item Taylorreeksen zijn machtreeksen. 
		\item Taylor en machtreeksen generaliseren polynomen: 
		\[
			p(x) = \sum_{k=0}^{n} a_k x^k = \sum_{k=0}^{n} b_k (x-x_0)^k
		\]
		\item Machtreeksen zijn eenvoudig te hanteren (bijvoorbeeld product): 
		\[
			(\sum a_k x^k)(\sum b_l x^l) = \sum c_m x^m
		\]
	\end{itemize}
	
	\newpage
	\section*{College 13}	
	
	\subsection*{Convergentiestraal van een machtreeks}
	\Def Zij $(b_n)$ een rij getallen, V de verzameling van haar verdichtingspunten. Definieer:
	\[
		\limsup_{n \to \infty} b_n = \begin{cases}
				\infty \quad &b_n \text{ onbegrensd naar boven} \\
				sup V \quad &\text{als } V = \emptyset \text{ onbegrensd naar boven}\\
				- \infty \quad &\text{als } V = \emptyset \text{ onbegrensd naar boven}
		\end{cases}		
	\]
	
	\Stel Als $\limsup b_n = L \in \reals$ dan 
	\begin{itemize}
		\item $\exists$ deelrij $b_{n_k}$ met $b_{n_k} \to L$. 
		\item $\forall_{\epsilon > 0} \exists_{n_0 \in N} \forall_{n \ge n_0}: b_n < L + \epsilon$ 
	\end{itemize}
	
	\Stel[Converentiestraal van een MR] Zij * een machtreeks. 
	\begin{itemize}
		\item Als $\limsup_{k \to \infty} \sqrt[k]{|ak|} = 0$ dan is * absoluut convergent voor alle $x \in \reals$. 
		\item Als $\limsup_{k \to \infty} \sqrt[k]{|ak|} = L \in (0,\infty)$ dan is * absoluut convergent voor alle $x \in (x_0 - R, x_0 + R)$ met $R = \frac{1}{\limsup \sqrt[k]{|ak|}}$ en divergent als $|x - x_0| > R$. 
		\item Als $\limsup_{k \to \infty} \sqrt[k]{|ak|} = \infty$ dan convergeert allen voor $x = x_0$. 
	\end{itemize}
	
	\Def $R = \frac{1}{\limsup \sqrt[k]{|ak|}}$ heet de convergentiestraal van *. 
	
	\Stel[Alternatief voor bepalen van convergentiestraal] Zij * een machtreeks. \begin{itemize}
		\item Als $\lim_{k \to \infty} \left| \frac{a_{k+1}}{a_k} \right| = 0$ dan $R = \infty$. 
		\item Als $\lim_{k \to \infty} \left| \frac{a_{k+1}}{a_k} \right| = L \in (0,\infty)$ dan $R = L$. 
		\item Als $\lim_{k \to \infty} \left| \frac{a_{k+1}}{a_k} \right| = \infty$ dan $R = 0$. 
	\end{itemize}
	
	\newpage
	\section*{College 14}
	
	\subsection*{Rekenen met machtreeksen}
	\Stel[Uniforme convergentie van machtreeksen] Zij * een machtreeks met convergentie straal $R > 0$. Zij $r \in (0,R)$. Dan convergeert * uniform op $[x_0 - r, x_0 + r]$. 
	
	\Stel[Machtreeksen stellen continue functies voor] De machtreeks * defini\"eert een continue functie $f: (x_0 - R, x_0 + R) \to \reals$. Dan $f(x) = \sum_{k = 0}^{\infty} a_k (x - x_0)^k$. 
	
	\Stel De machtreeksen $\sum_{k = 0}^{\infty} a_k (x - x_0)^k$, $\sum_{k = 0}^{\infty}k \cdot  a_k (x - x_0)^{k-1}$ en $\sum_{k = 0}^{\infty} k \cdot (k-1) \cdot a_k (x - x_0)^{k-2}$ hebben dezelfde convergentiestraal. 
	
	\Stel Zij $\sum_{k = 0}^{\infty} a_k (x - x_0)^k$ een machtreeks met convergentiestraal $R > 0$. 
	\begin{itemize}
		\item De functie $f: (x_0 - R, x_0 + R) \to \reals$ gegeven door $f(x) = \sum_{k = 0}^{\infty} a_k (x - x_0)^k$ is willekeurig vaak differentieerbaar op $(x_0 - R, x_0 + R)$ en 
		\[
			f^{(j)} (x) = \sum_{k = 0}^{\infty} k \cdot (k-1) \cdots (k - j + 1) a_k (x - x_0)^k
		\]
		Deze functie heeft convergentiestraal $R$. 
		\item De functie $F : (x_0 - R, x_0 + R) \to \reals$ gegeven door $\sum_{k = 0}^{\infty} \frac{a_k}{k+1} (x - x_0)^{k+1}$ is differentieerbaar op $(x_0 - R, x_0 + R)$ en $F'(x) = f(x)$. 
 	\end{itemize}
 	
 	\Stel[Identiteitsstelling] $\sum_{k = 0}^{\infty} a_k (x - x_0)^k$ en $\sum_{k = 0}^{\infty} b_k (x - x_0)^k$ zijn over machtreeksen met positieve convergentiestralen $R_1$ en $R_2$ met $R_1 \le R_2$ en 
 	\[
	 	\sum_{k = 0}^{\infty} a_k (x - x_0)^k = \sum_{k = 0}^{\infty} b_k (x - x_0)^k \quad \forall_{x \in (x_0 - \epsilon, x_0 + \epsilon)}.
 	\]
 	Dan $a_k = b_k \quad \forall_{k \in \naturals}$ en $R_1 = R_2$. 
 	
 	\Stel Zij $\sum_{k = 0}^{\infty} a_k (x - x_0)^k$ een machtreeks met positieve convergentiestraal. Defini\"eer $f(x) = \sum_{k = 0}^{\infty} a_k (x - x_0)^k$. Dan is de Taylorreeks van $f$ gegeven door $\sum_{k = 0}^{\infty} a_k (x - x_0)^k$. 1
	
	
	
\end{document}