% !TeX spellcheck = nl_NL
\documentclass[week=1]{homework}

\date{\today}

\begin{document}
    \maketitle
    \thispagestyle{empty}
    \newpage
    \begin{questions}
		\let\firstquestion\question
		\renewcommand*{\question}{\vspace{7mm}\firstquestion}
        %%%%%%%%%%%%
        % QUESTION 1
        %%%%%%%%%%%%
        \firstquestion
        
        Zijn de volgende deelverzamelingen van $\reals$ begrensd naar boven / beneden?
        Bepaal supremum, inmum, maximum en minimum (voor zover die bestaan).
        
        \begin{align*}
        	A &= \{ x \in \reals \mid \exists_{n \in \naturals} \colon 2n-1 < x < 2n \} \\
        	B &= \left\{ - \frac{1}{n} \ \middle| \ n \in \naturals_+ \right\} \\
        	C &= \{ x \in \reals \mid 4x - x^2 > 3 \} \\
        	D &= \{ x \in \reals \mid 4x - x^2 \geq 3 \} \\
        	E &= [0,1]\backslash \rationals
        \end{align*}
        
        % INSERT Q1
        
        %%%%%%%%%%%%
        % QUESTION 2
        %%%%%%%%%%%%
        \question
        Zij $A \subset \reals$ niet leeg en begrensd, $\epsilon > 0$. Laat zien: Er is een $x \in A$ zodanig dat $x > \sup A - \epsilon$ en er is een $y \in A$ zodanig dat $y < \inf A + \epsilon$.
        
        (\textbf{Hint:} Geef een bewijs uit het ongerijmde!)
        
        % INSERT Q2
        
        %%%%%%%%%%%%
        % QUESTION 3
        %%%%%%%%%%%%
        \question
        
        Zij $A,B \subset \reals$ niet leeg en begrensd. Definieer:
        \begin{align*}
	        -A &= \{-a \mid a \in A\} \\
		    A + B &= \{a + b \mid a \in A, b \in B\} \\
			A - B &= \{a - b \mid a \in A, b \in B\}
        \end{align*}
        
        Laat zien
        \begin{parts}
        	\part  
	        	\begin{align*}
	        		\sup (-A) &= - \inf A \\
	        		\sup (A+B) &= \sup A + \sup B \\
	        		\inf (A-B) &= \inf A - \sup B
	        	\end{align*}
	        	
	        	\begin{toprove}
	        		$\sup (-A) = - \inf A$.
	        	\end{toprove}
	        	\begin{proof}
	        		$A$ is begrensd en niet leeg, dus $-A$ ook niet begrensd (per definitie van de verzameling $-A$) en niet leeg. Dus $\sup(-A)$ bestaat.
	        		
	        		Per definitie van $\inf A$ weten we dat $\inf A$ de grootste ondergrens is waarvoor geldt
	        		\[
		        		\forall_{x \in A}: x \geq \inf A \,,
		        	\]
		        	maar dan ook
	        		\[
		        		\forall_{x \in A}: -x \leq -\inf A \,.
	        		\]
	        		
	        		Hieruit volgt dat $-\inf A$ een bovengrens is voor $-A$, ofwel $-\inf A \geq \sup(-A)$. Het rest nog om aan te tonen dat $-\inf A$ de kleinste bovengrens is, dus
	        		\[
		        		\forall_{\epsilon > 0}\exists_{x \in -A}: x > -\inf A - \epsilon.
	        		\]
	        		
	        		Maar we weten dat $\inf A$ de grootste bovengrens is voor $A$, dus
	        		\[
		        		\forall_{\epsilon > 0}\exists_{x \in A}: x < \inf A + \epsilon\,,
	        		\]
	        		Waar---door beide kanten met $-1$ te vermenigvuldigen---gebruikmakend het feit dat voor $x\in A$ geldt $-x \in -A$ de bewering direct uit volgt, dus
	        		\[
		        		\sup (-A) = - \inf A.
	        		\]
	        	\end{proof}
	        	
	        	\begin{toprove}
	        		$\sup (A+B) = \sup A + \sup B$
	        	\end{toprove}
	        	\begin{proof}
	        		$A$ en $B$ zijn begrensd en niet leeg, dus $A+B$ ook niet begrensd (per definitie van de verzameling $A+B$) en niet leeg. Dus $\sup(A+B)$ bestaat.
	        		
	        		Voor $\sup A$ en $\sup B$ weten we dat geldt
	        		\begin{align*}
	        			\forall_{x \in A}: x &\leq \sup A \\
	        			\forall_{y \in B}: y &\leq \sup B\,. \\
	        		\end{align*}
	        		
	        		Maar dan ook $\forall_{x\in A}\forall_{y \in B}: x + y \leq \sup A + \sup B$, dus $\sup A + \sup B$ bovengrens voor $A+B$ en $\sup A + \sup B \geq \sup(A+B)$. 
	        		
	        		We moeten enkel nog aantonen dat $\sup A + \sup B$ de kleinste bovengrens is voor $A+B$, dus
	        		\[
		        		\forall_{\epsilon > 0}\exists_{z \in A+B}: z > \sup A + \sup B - \epsilon\,.
	        		\]
	        		
	        		Zij $\epsilon > 0$, we weten
	        		\begin{align*}
	        			\forall_{\epsilon_1 > 0}\exists_{\hat x \in A}: \hat x &> \sup A - \epsilon_1 \\
	        			\forall_{\epsilon_2 > 0}\exists_{\hat y \in B}: \hat y &> \sup B - \epsilon_2 \,. \\
	        		\end{align*}
	        		Kies $\epsilon_1 = \epsilon_2 = \frac{\epsilon}{2}$, dan---aangezien $\hat x + \hat y \in A+B$--- weten we
			        \begin{align*}
			        	\exists_{z \in A+B}: z &= \hat x + \hat y > \sup A + \sup B - \epsilon_1 - \epsilon_2 \\
			        	\Leftrightarrow z &> \sup A + \sup B - \epsilon.
			        \end{align*}
			        
			        Dus $\sup (A+B) = \sup A + \sup B$.
	        	\end{proof}
        	
	        	\begin{toprove}
	        		$\inf (A-B) = \inf A - \sup B$
	        	\end{toprove}
	        	\begin{proof}
	        		$A$ en $B$ zijn begrensd en niet leeg, dus $A-B$ ook niet begrensd (per definitie van de verzameling $A-B$) en niet leeg. Dus $\inf(A-B)$ bestaat.
	        		
	        		Voor $\inf A$ en $\sup B$ weten we dat geldt
	        		\begin{align*}
	        		\forall_{x \in A}: x &\geq \inf A \\
	        		\forall_{y \in B}: y &\leq \sup B \\
	        		\Leftrightarrow -y &\geq -\sup B\,.
	        		\end{align*}
	        		
	        		Maar dan ook $\forall_{x\in A}\forall_{y \in B}: x - y \geq \inf A - \sup B$, dus $\inf A - \sup B$ ondergrens voor $A-B$ en $\inf A - \sup B \leq \inf(A-B)$. 
	        		
	        		We moeten enkel nog aantonen dat $\inf A - \sup B$ de grootste ondergrens is voor $A-B$, dus
	        		\[
		        		\forall_{\epsilon > 0}\exists_{z \in A-B}: z < \inf A - \sup B + \epsilon\,.
	        		\]
	        		
	        		Zij $\epsilon > 0$, we weten
	        		\begin{align*}
		        		\forall_{\epsilon_1 > 0}\exists_{\hat x \in A}: \hat x &< \inf A + \epsilon_1 \\
		        		\forall_{\epsilon_2 > 0}\exists_{\hat y \in B}: \hat y &> \sup B - \epsilon_2 \\
		        		\Leftrightarrow -\hat y &< -\sup B + \epsilon_2\,.
	        		\end{align*}
	        		Kies $\epsilon_1 = \epsilon_2 = \frac{\epsilon}{2}$, dan---aangezien $\hat x - \hat y \in A-B$--- weten we
	        		\begin{align*}
		        		\exists_{z \in A-B}: z &= \hat x - \hat y < \inf A - \sup B + \epsilon_1 + \epsilon_2 \\
		        		\Leftrightarrow z &< \inf A - \sup B + \epsilon\,.
	        		\end{align*}
	        		
	        		Dus $\inf (A-B) = \inf A - \sup B$.
	        	\end{proof}
	        \part 
		        \begin{toprove}
		        	$(\forall_{a \in A} \forall_{b \in B}: a < b) \implies \sup A \le \inf B$
		        \end{toprove}
		        \begin{proof}
		        	We bewijzen uit het ongerijmde en nemen aan $\forall_{a \in A} \forall_{b \in B}: a < b$.
		        	
		        	$A$ en $B$ zijn begrensd en niet leeg dus hebben beide infima en suprema. Er moet dus gelden $\sup A \le \inf B$ of $\sup A > \inf B$. we nemen aan $\sup A > \inf B$ en zien dat dit tot een tegenspraak leidt.
		        	
		        	We weten $\forall_{a\in A}: a \leq \sup A$ en $\forall_{b\in B}: b \geq \inf B$. Aangezien $\sup A > \inf B$ is er een $\epsilon_1 > 0$ zodat $\sup A - \epsilon_1 = \inf B$, maar dan
		        	\[
			        	\exists_{\hat a \in A}: \hat a > \sup A - \epsilon_1 = \inf B\,.
		        	\]
		        	We vinden $\inf B < \hat a$. Nu is er een $\epsilon_2 > 0$ zodat $\inf B + \epsilon_2 = \hat a$, dus
		        	\[
			        	\exists_{\hat b \in B}: \hat b < \inf B + \epsilon_2 = \hat a\,,
		        	\]
		        	waaruit volgt $\hat b < \hat a$, wat in tegenspraak is met $a < b$ voor alle $a \in A$ en $b \in B$. $\lightning$
		        	
		        	Dus $(\forall_{a \in A} \forall_{b \in B}: a < b) \implies \sup A \le \inf B$.
		        \end{proof}
        \end{parts}
     \end{questions}
\end{document}