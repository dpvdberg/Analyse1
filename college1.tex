% !TeX spellcheck = en_EN
\documentclass[week=1]{homework}

\date{\today}

\begin{document}
    \maketitle
    \thispagestyle{empty}
    \newpage
    \begin{questions}
		\let\firstquestion\question
		\renewcommand*{\question}{\vspace{7mm}\firstquestion}
        %%%%%%%%%%%%
        % QUESTION 1
        %%%%%%%%%%%%
        \firstquestion
        
        Zijn de volgende deelverzamelingen van $\reals$ begrensd naar boven / beneden?
        Bepaal supremum, inmum, maximum en minimum (voorzover die bestaan).
        
        \begin{align*}
        	A &= \{ x \in \reals \mid \exists_{n \in \naturals} \colon 2n-1 < x < 2n \} \\
        	B &= \left\{ - \frac{1}{n} \ \middle| \ n \in \naturals_+ \right\} \\
        	C &= \{ x \in \reals \mid 4x - x^2 > 3 \} \\
        	D &= \{ x \in \reals \mid 4x - x^2 \geq 3 \} \\
        	E &= [0,1]\backslash \rationals
        \end{align*}
        
        \[
	        A = \{ x \in \reals \mid \exists_{n \in \naturals} \colon 2n-1 < x < 2n \}
        \]
		Er geldt voor $n=0$ dat $-1 < x < 0$ (dus $-1 \not \in A$) en voor $n=1$ dat $1 < x < 2$, enzovoorts.
		      \begin{itemize}
		      	\item begrensd naar boven: nee
		      	\item begrensd naar beneden: ja 
		      	\item supremum: geen
		      	\item infimum: -1
		      	\item maximum: geen
		      	\item minimum: geen
		      \end{itemize}
		              
        \[
	        B = \left\{ - \frac{1}{n} \ \middle| \ n \in \naturals_+ \right\}
        \]
        We weten $\lim_{n \to \infty} -\frac{1}{n} = 0$, maar $0 \not \in B$. Deze verzameling kan in feite ook herschreven worden als:
        \[
	        B = \{x \in \reals \mid -1 \le x < 0 \}
        \]
        Het is nu duidelijk dat:
              \begin{itemize}
              	\item begrensd: ja
              	\item supremum: 0
              	\item infimum: -1
              	\item maximum: geen
              	\item minimum: -1
              \end{itemize}
              
        \[
	        C = \{ x \in \reals \mid 4x - x^2 > 3 \}
        \]
        Uitwerken van $4x - x^2 = 3$ geeft $x = 1$ en $x = 3$. Er volgt dat voor $x \in (1,3)$ geldt dat $4x - x^2 > 3$ en dat de verzameling herschreven worden als:
        \[
	        C = \{x \in \reals \mid 1 < x < 3 \}
        \]
        Het is nu duidelijk dat:              
              \begin{itemize}
              	\item begrensd: ja
              	\item supremum: 3
              	\item infimum: 1
              	\item maximum: geen
              	\item minimum: geen
              \end{itemize}
        
        \[
	        D = \{ x \in \reals \mid 4x - x^2 \geq 3 \}
        \]
        Dit gaat analoog aan de verzameling $C$. De verzameling $D$ kan herschreven worden als:
        \[
	        D = \{x \in \reals \mid 1 \leq x \leq 3 \}
        \]    
        Het is nu duidelijk dat:
              \begin{itemize}
              	\item begrensd: ja
              	\item supremum: 3
              	\item infimum: 1
              	\item maximum: 3
              	\item minimum: 1
              \end{itemize}
          
        \[
	        E = [0,1]\backslash \rationals
        \]    
        Deze verzameling kan herschreven worden als:
        \[
	        E = \{x \in \reals \mid 0 \leq x \leq 1 \wedge x \not\in \rationals \}
        \]
        
        Uit het feit dat de rationale getallen \textit{`dicht liggen'} vinden we
              \begin{itemize}
              	\item begrensd: ja
              	\item supremum: 1
              	\item infimum: 0
              	\item maximum: geen
              	\item minimum: geen
              \end{itemize}
                     
        %%%%%%%%%%%%
        % QUESTION 2
        %%%%%%%%%%%%
        \question
        Zij $A \subset \reals$ neit leeg en begrensd, $\epsilon > 0$. Laat zien: Er is een $x \in A$ zodanig dat $x > \sup A - \epsilon$ en er is een $y \in A$ zodanig dat $y < \inf A + \epsilon$.
        
        (\textbf{Hint:} Geef een bewijs uit het ongerijmde!)
        
        \toprove $\exists x \in A: x > \sup A - \epsilon$
               
        \proof 
        Stel dat de bewering niet waar is. Dan:
        \begin{align*}
        	\neg(&\exists x \in A: x > \sup A - \epsilon) \\
        	&\forall x \in A: x \le \sup A - \epsilon 
        \end{align*}
        Maar dan is $\sup A - \epsilon$ een kleinere bovengrens van $A$ dan $\sup A$, want $\epsilon > 0$, dus is $\sup A$ niet het supremum van $A$. $\lightning$
        
        Dan moet dus gelden dat: $\exists x \in A: x > \sup A - \epsilon$
        \qed
        
        \toprove $\exists y \in A: y < \inf A + \epsilon$
        
        \proof 
        Stel dat de bewering niet waar is. Dan: 
        \begin{align*}
        	\neg(&\exists x \in A: y < \inf A + \epsilon) \\
        	&\forall x \in A: y \ge \inf A + \epsilon
        \end{align*}
        Maar dan is $\inf A + \epsilon$ een grotere ondergrens van $A$ dan $\inf A$, want $\epsilon > 0$, dus is $\inf A$ niet het infimum van A. $\lightning$
        
        Dan moet dus gelden dat: $\exists y \in A: y < \inf A + \epsilon$
        
        \qed
        
        %%%%%%%%%%%%
        % QUESTION 3
        %%%%%%%%%%%%
        \question
        
        Zij $A,B \subset \reals$ niet leeg en begrensd. Definieer:
        \begin{align*}
	        -A &= \{-a \mid a \in A\} \\
		    A + B &= \{a + b \mid a \in A, b \in B\} \\
			A - B &= \{a - b \mid a \in A, b \in B\}
        \end{align*}
        
        Laat zien
        \begin{parts}
        	\part  
	        	\begin{align*}
	        		\sup (-A) &= - \inf A \\
	        		\sup (A+B) &= \sup A + \sup B \\
	        		\inf (A-B) &= \inf A - \sup B
	        	\end{align*}
	        	
	        	% INSERT Q3a
	        	
	        \part 
		        \[
		        (\forall a \in A \forall b \in B: a < b) \rightarrow \sup A \le inf B
		        \]
		        
		        % INSERT Q3b
        \end{parts}
     \end{questions}
\end{document}