% !TeX spellcheck = en_US
\documentclass[week=2]{homework}

\date{\today}

\begin{document}
    \maketitle
    \thispagestyle{empty}
    \newpage
    \begin{questions}
		\let\firstquestion\question
		\renewcommand*{\question}{\vspace{7mm}\firstquestion}
        %%%%%%%%%%%%
        % QUESTION 1
        %%%%%%%%%%%%
        \firstquestion
        Assuming $a,b,c \in \reals$ arbitrary, prove the following statements;
        \begin{parts}
        	\part \label{1:a}
        	\begin{inlinetoprove}
        		$|a+b| \le |a| + |b|$.
        	\end{inlinetoprove}
	        \begin{proof}
	        	For all $x \in \reals$, we can define $|x| = \max\{x, -x\}$. It follows that $\pm x \le |x|$.
	        	
	        	For $|a+b|$, we distinguish the two cases
	        	\begin{align*}
	        		|a+b| &= a + b \le |a| + b \le |a| + |b|\,, \text{or}\\
	        		|a+b| &= - (a + b) = - a - b \le |a| - b \le |a| + |b|\,.
	        	\end{align*}
	        	
	        	In any case we find
	        	\[
		        	|a+b| \le |a| + |b|\,.
	        	\]
	        \end{proof}
        
	        \part \label{1:b}
	        \begin{inlinetoprove}
	        	$|a-b| \le |a-c| + |b-c|$.
	        \end{inlinetoprove}
	        \begin{proof}
	        	Using \ref{1:a}, we find $|x+y| \le |x| + |y|$ for all $x,y \in \reals$. Let $x = a-c$ and $y = c-b$, then
	        	\[
		        	|x+y| = |a-b| = |a-c+c-b| \le |a-c| + |b-c|\,.
	        	\]
	        \end{proof}
        
	        \part \label{1:c}
	        \begin{inlinetoprove}
	        	$||a| - |b|| \le |a-b|$.
	        \end{inlinetoprove}
	        \begin{proof}
	        	Using \ref{1:a} we find
	        	\begin{align*}
		        	|b| = |b-a+a| &\le |b-a| + |a| \\
		        	\therefore |b| - |a| &\le |b-a|\,,
		        	\intertext{Using similar reasoning we find}
		        	|a| - |b| &\le |a-b|\,.
	        	\end{align*}
	        	But $|a-b| = |-(a-b)| = |b-a|$, hence:
	        	\begin{equation} \label{Q1:1}
	        	\begin{split}
		        	|a-b| &\ge |a| - |b|\,, \text{ and} \\
		        	|a-b| &\ge |b| - |a|	
	        	\end{split}
	        	\end{equation}
	        	
	        	Now remember that:
		        \begin{equation} \label{Q1:2}
		        	||a| - |b|| = \max \{|a| - |b|, - (|a| - |b|) \} = \max \{|a| - |b|, - |a| + |b| \}		        	
		        \end{equation}
		        
	        	Using\ref{Q1:1} and \ref{Q1:2}, we can now conclude:
	        	\[
		        	|a-b| \ge ||a|-|b||\,.
	        	\]
	        \end{proof}
        \end{parts}
    
	    \question
	    Let $\{a_n\}$ be a sequence, $a^* \in \reals$.
	    
	    \begin{inlinetoprove}
	    	$a_n \to a^* \iff |a_n - a^*| \to 0$.
	    \end{inlinetoprove}
	    \begin{proof}
	    	% INSERT Q2
	    \end{proof}
    
	    \question
	    Let $\{a_n\}$ and $\{b_n\}$ be sequences with limits $a_n \to a^*$, $b_n \to b^*$.
	    Proof or contradict the following statement;
	    \[
		    \forall_{n \in \naturals}: a_n < b_n \implies a^* < b^*
	    \]
	    
	    % INSERT Q3
	    
	    \question
	    Let $\{a_n\}$, $\{b_n\}$ and $\{c_n\}$ be sequences defined by
	    \[
		    a_n = \frac{n+3}{n^2-3}\,,\qquad b_n = \frac{n^5}{3^n}\,,\qquad c_n = \frac{7n+4}{n+5}\,.
	    \]
	    
	    Find the limits of these sequences and prove the convergence of these sequences (without applying limit theorems).
	    \begin{parts}
	    	\part
	    	% INSERT Q4a
	    	
	    	\part 
	    	% INSERT Q4b
	    	
	    	\part
	    	% INSERT Q4c
	    \end{parts}
     \end{questions}
\end{document}