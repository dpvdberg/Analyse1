% !TeX spellcheck = en_US
\documentclass[week=5]{homework}

\date{\today}


\begin{document}
    \maketitle
    \thispagestyle{empty}
    \newpage
    \begin{questions}
		\let\firstquestion\question
		\renewcommand*{\question}{\vspace{7mm}\firstquestion}
        \firstquestion
        A \textit{telescoping series} is a sum $\sum_{k=1}^{\infty} a_k$ where $a_k = b_k - b_{k-1}$, $k = 1, 2, \ldots$.
        
        \begin{parts}
        	\part \label{1:A}
        	\begin{inlinetoprove}
        		A telescoping series $\sum_{k=1}^{\infty} a_k$ is convergent if and only if $b_n$ is convergent and in this case we have
        		\[
	        		\sum_{k=1}^{\infty}a_k = \lim_{n\to\infty} b_n - b_0
        		\]
        		\begin{proof}
        			% INSERT 1A
        		\end{proof}
        	\end{inlinetoprove}
        
	        \part 
	        Using \ref{1:A}, find the value of
	        \[
		        \sum_{k=2}^{\infty} \frac{2k+1}{k^2(k+1)^2}
	        \]
        \end{parts}
    
	    \question
	    Determine whether the given series converges or diverges. In the case of a convergence, find the sum.
	    
	    
	    \begin{enumerate}[label=(\alph*)]
	    	\item Define $a_k = \frac{(-1)^k}{2^{k-1}}$. We are interested in 
	    	\[
		    	\sum_{k=2}^{\infty} a_k = \sum_{k=2}^{\infty} \frac{(-1)^k}{2^{k-1}}\,.
	    	\]
	    	
	    	\item Define $a_k = \frac{3^{k-1} - 2^k}{6^k}$. We are interested in 
	    	\[
	    	\sum_{k=1}^{\infty} a_k = \sum_{k=1}^{\infty} \frac{3^{k-1} - 2^k}{6^k}\,.
	    	\]
	    	
	    	\item Define $a_k = \frac{1}{k^2 + 3k + 2}$. We are interested in 
	    	\[
	    	\sum_{k=1}^{\infty} a_k = \sum_{k=1}^{\infty} \frac{1}{k^2 + 3k + 2}\,.
	    	\]
	    	
	    	\addtocounter{enumi}{1}
	    	\item Define $a_k = \frac{k}{2k+1}$. We are interested in 
	    	\[
	    	\sum_{k=1}^{\infty} a_k = \sum_{k=1}^{\infty} \frac{k}{2k+1}\,.
	    	\]
	    	
	    	\item Define $a_k = \frac{2}{k^2+2k}$. We are interested in 
	    	\[
	    	\sum_{k=1}^{\infty} a_k = \sum_{k=1}^{\infty} \frac{2}{k^2+2k}\,.
	    	\] 
	    	
	    	\item Define $a_k = \ln \frac{1}{k}$. We are interested in 
	    	\[
	    	\sum_{k=1}^{\infty} a_k = \sum_{k=1}^{\infty} \ln \frac{1}{k}\,.
	    	\] 
	    	
	    	\addtocounter{enumi}{1}
	    	\item Define $a_k = \sqrt{k^2 + k} -k$. We are interested in 
	    	\[
	    	\sum_{k=1}^{\infty} a_k = \sum_{k=1}^{\infty} \sqrt{k^2 + k} -k\,.
	    	\]
	    	
	    	\item Define $a_k = k \sin \frac{1}{k}$. We are interested in 
	    	\[
	    	\sum_{k=1}^{\infty} a_k = \sum_{k=1}^{\infty} k \sin \frac{1}{k}\,.
	    	\] 
	    	
	    	\addtocounter{enumi}{1}
	    	\item Define $a_k = \left(\frac{k-2}{k}\right)^k$. We are interested in 
	    	\[
	    	\sum_{k=1}^{\infty} a_k = \sum_{k=1}^{\infty} \left(\frac{k-2}{k}\right)^k\,.
	    	\]
	    \end{enumerate}
    
	    \question
	    Let $\sum a_n$, $\sum b_n$ be two series, $c\in\reals$. Proof or disprove the following statements.
	    
	    \begin{parts}
	    	\part
	    	If $\sum a_n$ and $\sum b_n$ are convergent, then so is $\sum (a_n + b_n)$ and $\sum (a_n + b_n) = \sum a_n + \sum b_n$.
	    	
	    	\begin{proof}
	    		content...
	    	\end{proof}
    	
	    	\part
	    	If $\sum a_n$ is convergent then so is $\sum ca_n$ and $\sum ca_n = c\sum a_n$.
	    	
	    	\begin{proof}
	    		content...
	    	\end{proof}
    	
	    	\part
	    	If $\sum a_n$ is convergent and $\sum b_n$ is divergent, then $\sum (a_n + b_n)$ is divergent.
	    	
	    	\begin{proof}
	    		content...
	    	\end{proof}
	    	
	    	\part
	    	If $\sum a_n$ and $\sum b_n$ are divergent, then so is $\sum (a_n + b_n)$.
	    	
	    	%PROOF BY CONTRADICTION a_n = 1, b_n = -1 bijv.
	    \end{parts}
    
	    \question
	    Let $\sum_{k=1}^{\infty} a_k$ be an absolutely converging series.
	    
	    \begin{toprove}
	    	\[
		    	\left| \sum_{k=1}^{\infty} a_k \right| \leq \sum_{k=1}^{\infty} |a_k|
	    	\]
	    \end{toprove}
    
	    \begin{proof}
	    	content...
	    \end{proof}
    
	    \question
	    Let $q \in \reals$, $|q| < 1$. Determine the values of the following series;
	    
	    \begin{parts}
	    	\part Define $a_k = \frac{q^k}{2}$. We are interested in 
	    	\[
	    	\sum_{k=7}^{\infty} a_k = \sum_{k=7}^{\infty} \frac{q^k}{2}\,.
	    	\]
	    	
	    	We have
	    	\begin{align*}
	    		\sum_{k=7}^{\infty} \frac{q^k}{2} &= \sum_{k=0}^{\infty} \frac{q^k}{2} - \sum_{k=0}^{6} \frac{q^k}{2} \\
	    		&= \frac{1}{2} \cdot \sum_{k=0}^{\infty} \frac{q^k}{2} - \sum_{k=0}^{6} \frac{q^k}{2} \\
	    		&= \frac{1}{2} \cdot \frac{1}{1-q} - \frac{1}{2} \cdot \frac{1 - q^7}{1-q} \\
	    		&= \frac{q^7}{2-2q}
	    	\end{align*}
	    	
	    	\part Define $a_k = \left(\frac{q}{2}\right)^k$. We are interested in 
	    	\[
	    	\sum_{k=7}^{\infty} a_k = \sum_{k=7}^{\infty} \left(\frac{q}{2}\right)^k\,.
	    	\]
	    	
	    	We have 
	    	\begin{align*}
	    		\sum_{k=7}^{\infty} \left(\frac{q}{2} \right)^k &= \sum_{k=0}^{\infty} \left(\frac{q}{2} \right)^k - \sum_{k=0}^{6} \left(\frac{q}{2} \right)^k \\
	    		&= \frac{1}{1 - \frac{q}{2}} - \frac{1 - \left(\frac{q}{2} \right)^7}{1 - \frac{q}{2}} \\
	    		&= \frac{q^7}{2^7 - 2^6 q}
	    	\end{align*}
	    	
	    	\part Define $a_k = \frac{2^{k/2}}{2^{2k}}$. We are interested in 
	    	\[
	    	\sum_{k=0}^{\infty} a_k = \sum_{k=0}^{\infty} \frac{2^{k/2}}{2^{2k}}\,.
	    	\]
	    	
	    	We have
	    	\begin{align*}
	    		\sum_{k=0}^{\infty} \frac{2^{\frac{k}{2}}}{2^{2k}} &= \sum_{k=0}^{\infty} \left(\frac{1}{2\sqrt{2}} \right)^k \\
	    		&= \frac{1}{1 - \frac{1}{2\sqrt{2}}} \\
	    		&= \frac{2}{7}\left(4 + \sqrt 2 \right)\,.
	    	\end{align*}
	    	
	    	\part Define $a_k = \frac{4^{k-5}}{5^{2k+2}}$. We are interested in 
	    	\[
	    	\sum_{k=0}^{\infty} a_k = \sum_{k=0}^{\infty} \frac{4^{k-5}}{5^{2k+2}}\,.
	    	\]
	    	
	    	We have
	    	\begin{align*}
	    		\sum_{k=0}^{\infty} \frac{4^{k - 5}}{5^{2k + 2}} &= \sum_{k=0}^{\infty} \frac{4^{-5}}{5^2} \cdot \sum_{k=0}^{\infty} \left(\frac{4}{25} \right)^k \\
	    		&= \frac{1}{4^5 \cdot 5^2} \cdot \frac{1}{1 - \frac{4}{25}} \\
	    		&= \frac{1}{25600} \cdot \frac{25}{25 - 4} \\
	    		&= \frac{1}{21504}\,.
	    	\end{align*}
	    	
	    \end{parts}
     \end{questions}
\end{document}