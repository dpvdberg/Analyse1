% !TeX spellcheck = en_US
\documentclass[week=5]{homework}

\date{\today}


\begin{document}
    \maketitle
    \thispagestyle{empty}
    \newpage
    \begin{questions}
		\let\firstquestion\question
		\renewcommand*{\question}{\vspace{7mm}\firstquestion}
        \firstquestion
        A \textit{telescoping series} is a sum $\sum_{k=1}^{\infty} a_k$ where $a_k = b_k - b_{k-1}$, $k = 1, 2, \ldots$.
        
        \begin{parts}
        	\part \label{1:A}
        	\begin{inlinetoprove}
        		A telescoping series $\sum_{k=1}^{\infty} a_k$ is convergent if and only if $b_n$ is convergent and in this case we have
        		\[
	        		\sum_{k=1}^{\infty}a_k = \lim_{n\to\infty} b_n - b_0\,.
        		\]
        		\begin{proof}
        			We prove both sides of the implication
        			\begin{itemize}
        				\item $\impliedby$
        				
        				Because $\sum_{k=1}^{\infty} a_k$ is a telescoping series, the following holds for all partial sums
        				\begin{align*}
        				S_n = \sum_{k=1}^{n} a_k &= a_1 + a_2 + ... + a_{n-1} + a_n \\
        				&= b_1 - b_0 + b_2 - b_1 + ... + b_{n-1} - b_{n-2} + b_n - b_{n-1} \\
        				&= b_n - b_0\,.
        				\end{align*}
        				Now let $n \to \infty$, then $b_n \to b^* \in \reals$ and hence, $\lim_{n \to \infty} b_n - b_0 = b^* - b_0 = \bar b \in \reals$. We can conclude 
        				\[
	        				\sum_{k=1}^{\infty} a_k = \lim_{n \to \infty} \sum_{k=1}^{n} a_k = \lim_{n \to \infty} S_n = \lim_{n \to \infty} b_n - b_0 = \bar b \in \reals\,,
        				\]
        				which in turn means the sum converges.
        				
        				\item $\implies$
        				
        				Now $\sum_{k=1}^{\infty} a_k$ converges and again we find for some $a^* \in \reals$
	        			\begin{align*}
	        				a^* = \sum_{k=1}^{\infty} a_k = \lim_{n \to \infty} \sum_{k=1}^{n} a_k &= \lim_{n\to\infty} (b_n - b_0) = \lim_{n\to\infty} b_n - b_0\,.
	        			\end{align*}
	        			We find $\lim_{n\to\infty} b_n = a^* + b_0 = \bar a \in \reals$, which means the sequence $b_n$ converges.
        			\end{itemize}
        		\end{proof}
        	\end{inlinetoprove}
        
	        \part 
	        Using \ref{1:A}, find the value of
	        \[
		        \sum_{k=2}^{\infty} \frac{2k+1}{k^2(k+1)^2}
	        \]
	        
	        If we decompose the fraction, we get the following equation:
	        \[
		        \sum_{k=2}^{\infty} \frac{2k+1}{k^2(k+1)^2} = \sum_{k=2}^{\infty} \frac{1}{k^2} - \frac{1}{(k+1)^2}
	        \] 
	        
	        Define $b_k := 1/b^2$, the partial sums of this series have the following decompositions;
	        \begin{align*}
		        S_n = \sum_{k = 2}^{n} \frac{2k+1}{k^2(k+1)^2} &= \frac{1}{4} - \frac{1}{9} + \frac{1}{9} - \frac{1}{16} + ... + \frac{1}{(n - 1)^2} - \frac{1}{n^2} + \frac{1}{n^2} - \frac{1}{(n + 1)^2}\\
		        &= \frac{1}{4} - \frac{1}{(n + 1)^2}  \\
		        &= b_2 - b_{n+1}\,.
	        \end{align*}
	        Using \ref{1:A}, we can now conclude that the original sum converges because $b_{n+1}$ converges as $n \to \infty$. Now, we can calculate the following: 
	        \[
		        \sum_{k=2}^{\infty} \frac{2k+1}{k^2(k+1)^2} = \lim_{n \to \infty} S_n = \frac{1}{4} - \lim_{n \to \infty} \frac{1}{(n + 1)^2} = \frac{1}{4}\,.
	        \]
	        
        \end{parts}
    
	    \question
	    Determine whether the given series converges or diverges. In the case of a convergence, find the sum.
	    
	    
	    \begin{enumerate}[label=(\alph*)]
	    	\item Define $a_k = \frac{(-1)^k}{2^{k-1}}$. We are interested in 
	    	\[
		    	\sum_{k=2}^{\infty} a_k = \sum_{k=2}^{\infty} \frac{(-1)^k}{2^{k-1}}\,.
	    	\]
	    	We find that
	    	\begin{align*}
	    	\sum_{k = 2}^{\infty} \frac{(-1)^k}{2^{k-1}}&= \sum_{k = 1}^{\infty} \frac{(-1)^{k-1}}{2^{k-1}} \\
	    	&= \sum_{k = 1}^{\infty} \left(\frac{-1}{2} \right)^{k-1} \\
	    	&= - \sum_{k = 1}^{\infty} \left(\frac{-1}{2} \right)^{k} && \mbox{this is a standard geometric series}\\ 
	    	&= - \left(\frac{- \frac{1}{2}}{1 + \frac{1}{2}} \right)  \\
	    	&= \frac{1}{3}\,.
	    	\end{align*}
	    	
	    	\item Define $a_k = \frac{3^{k-1} - 2^k}{6^k}$. We are interested in 
	    	\[
	    	\sum_{k=1}^{\infty} a_k = \sum_{k=1}^{\infty} \frac{3^{k-1} - 2^k}{6^k}\,.
	    	\]
	    	We find that
	    	\begin{align*}
	    	\sum_{k = 1}^{\infty} \frac{3^{k-1} - 2^k}{6^k} &= \sum_{k = 1}^{\infty} \frac{1}{3} \cdot \left(\frac{1}{2} \right)^k - \left(\frac{1}{3} \right)^k \\
	    	&= \frac{1}{3} \sum_{k = 1}^{\infty} \left(\frac{1}{2} \right)^k - \sum_{k = 1}^{\infty} \left(\frac{1}{3} \right)^k && \mbox{these are both standard geometric series}\\
	    	&= \frac{1}{3} \cdot \frac{1/2}{1 - 1/2} - \frac{1/3}{1 - 1/3} \\
	    	&= - \frac{5}{6}\,.
	    	\end{align*}
	    	
	    	\item Define $a_k = \frac{1}{k^2 + 3k + 2}$. We are interested in 
	    	\[
	    	\sum_{k=1}^{\infty} a_k = \sum_{k=1}^{\infty} \frac{1}{k^2 + 3k + 2}\,.
	    	\]
	    	\[
	    	\sum_{k = 1}^{\infty} \frac{1}{k^2 + 3k + 2} = \sum_{k = 1}^{\infty} \frac{1}{k + 1} - \frac{1}{k + 2} 
	    	\]
	    	This is a telescope series, for which it holds that:
	    	\[
	    	S_n = \frac{1}{2} - \frac{1}{3} + \frac{1}{3} - \frac{1}{4} + \cdots + \frac{1}{n + 1} - \frac{1}{n + 2} = \frac{1}{2} - \frac{1}{n + 2}
	    	\]
	    	This converges to $1/2$ as $n \to \infty$.
	    	
	    	\addtocounter{enumi}{1}
	    	\item Define $a_k = \frac{k}{2k+1}$. We are interested in 
	    	\[
		    	\sum_{k=1}^{\infty} a_k = \sum_{k=1}^{\infty} \frac{k}{2k+1} = \sum_{k = 1}^{\infty} \frac{1}{2 + \frac{1}{k}}\,.
	    	\]
	    	But this means
	    	\[
		    	a_k = \frac{1}{2 + \frac{1}{k}} \xrightarrow{k\to\infty} \frac{1}{2} \neq 0\,.
	    	\]
	    	Therefore, the sum $\sum_{k = 1}^{\infty} \frac{k}{2k + 1}$ diverges, as a consequence of theorem $7.1.9$.
	    	
	    	\item Define $a_k = \frac{2}{k^2+2k}$. We are interested in 
	    	\[
	    	\sum_{k=1}^{\infty} a_k = \sum_{k=1}^{\infty} \frac{2}{k^2+2k}\,.
	    	\] 
	    	\[
	    	\sum_{k = 1}^{\infty} \frac{2}{k^2 + 2k} = \sum_{k = 1}^{\infty} \frac{1}{k} - \frac{1}{k + 2}
	    	\]
	    	This is a telescope series, for which it holds that:
	    	\[
	    	S_n =  1 - \frac{1}{3} + \frac{1}{2} - \frac{1}{4} + ... + \frac{1}{n - 1} - \frac{1}{n + 1} + \frac{1}{n} - \frac{1}{n + 2} = 1 + \frac{1}{2} - \frac{1}{n + 1} - \frac{1}{n + 2}
	    	\]
	    	This converges to $3/2$ as $n \to \infty$.
	    	
	    	\item Define $a_k = \ln \frac{1}{k}$. We are interested in 
	    	\[
	    	\sum_{k=1}^{\infty} a_k = \sum_{k=1}^{\infty} \ln \frac{1}{k}\,.
	    	\] 
	    	However, 
	    	\[
	    	\ln \frac{1}{k} \xrightarrow{k\to\infty} -\infty \neq 0\,.
	    	\]
	    	Therefore, the sum $\sum_{k = 1}^{\infty} \ln \frac{1}{k}$ diverges, as a consequence of theorem $7.1.9$.
	    	
	    	\addtocounter{enumi}{1}
	    	\item Define $a_k = \sqrt{k^2 + k} -k$. We are interested in 
	    	\[
	    	\sum_{k=1}^{\infty} a_k = \sum_{k=1}^{\infty} \left(\sqrt{k^2 + k} -k\right)\,.
	    	\]
	    	We have
	    	\begin{align*}
	    	a_k = \left(\sqrt{k^2 + k} - k \right) &= \frac{k}{\sqrt{k^2 + k} + k } \\
	    	&= \frac{1}{\sqrt{k + 1/k} + 1}\,.
	    	\end{align*}
	    	However,
	    	\[
	    	\frac{1}{\sqrt{k + 1/k} + 1} \xrightarrow{k\to\infty} \frac{1}{2} \neq 0
	    	\]
	    	Therefore, $\sum_{k = 1}^{\infty} \left(\sqrt{k^2 + k} - k \right)$ diverges, as a consequence of theorem $7.1.9$.	       	
	    	
	    	\addtocounter{enumi}{2}
	    	\item Define $a_k = \left(\frac{k-2}{k}\right)^k$. We are interested in 
	    	\[
	    	\sum_{k=1}^{\infty} a_k = \sum_{k=1}^{\infty} \left(\frac{k-2}{k}\right)^k = \sum_{k = 1}^{\infty} \left(1 - \frac{2}{k} \right)^k\,.
	    	\]
	    	However, using $u = -k/2$, we find
	    	\begin{align*}
		    	a_k = \left(1 - \frac{2}{k} \right)^k &= \left(1 + \frac{1}{u} \right)^{-2u} \\
		    	&= \left(\left(1 + \frac{1}{u} \right)^u\right)^{-2} \xrightarrow{u\to\infty} \frac{1}{e^2} \neq 0.
	    	\end{align*}
	    	Therefore, the sum $\sum_{k = 1}^{\infty} \left(\frac{k - 2}{k} \right)^k$ diverges, as a consequence of theorem $7.1.9$.
	    	
	    \end{enumerate}
    
	    \question
	    Let $\sum a_n$, $\sum b_n$ be two series, $c\in\reals$. Proof or disprove the following statements.
	    
	    \begin{parts}
	    	\part
	    	\begin{inlinetoprove}
	    		If $\sum a_n$ and $\sum b_n$ are convergent, then so is $\sum (a_n + b_n)$ and $\sum (a_n + b_n) = \sum a_n + \sum b_n$.
	    	\end{inlinetoprove}
	    	
	    	\begin{proof}
	    		We know that $\sum a_n$ and $\sum b_n$ are convergent. Let $S_n$ be the sequence of partial sums of $\sum a_n$, let $T_n$ be the partial sums of $\sum b_n$ and let $R_n$ be the partial sums of $\sum (a_n + b_n)$.
	    		By associativity of addition we have
	    		\begin{align*}
	    			\sum_{k=1}^{n} (a_n + b_n) &= \sum_{k=1}^{n} a_n + \sum_{k=1}^{n} b_n \\
	    			\therefore R_n &= S_n + T_n\,.
	    		\end{align*}
	    		
	    		$\sum a_n$ is convergent which implies convergence of $S_n$. The same holds for $T_n$.
	    		
	    		But for sequences, we have already proven that if the sequences $S_n$ and $T_n$ are convergent, then the sequence $R_n = S_n + T_n$ is also convergent and
	    		\[
		    		\sum (a_n + b_n) = \lim_{n \to \infty} R_n = \lim_{n \to \infty} S_n + \lim_{n \to \infty} T_n = \sum a_n + \sum b_n
	    		\]
	    		
	    	\end{proof}
    	
	    	\part
	    	\begin{inlinetoprove}
	    		If $\sum a_n$ is convergent then so is $\sum ca_n$ and $\sum ca_n = c\sum a_n$.
	    	\end{inlinetoprove}
	    	
	    	\begin{proof}
	    		Let $S_n$ be the sequence of partial sums of $\sum a_n$. Because $\sum a_n$ converges, we know that $S_n$ also converges. 
	    		
	    		For sequences, we know that if $\lim_{n \to \infty} S_n$ exists, then also $\lim_{n \to \infty} c \cdot S_n$ exists and
	    		\[
		    		\lim_{n\to\infty} c \cdot S_n = c\lim_{n\to\infty} S_n\,.
	    		\]
	    		
	    		We conclude
	    		\begin{align*}
	    			\sum c \cdot a_n &= \lim_{n \to \infty} c \cdot S_n \\
	    			&= c \cdot \lim_{n \to \infty} S_n \\
	    			&= c \cdot \sum a_n 
	    		\end{align*}
	    	\end{proof}
    	
	    	\part
	    	\begin{inlinetoprove}
	    		If $\sum a_n$ is convergent and $\sum b_n$ is divergent, then $\sum (a_n + b_n)$ is divergent.
	    	\end{inlinetoprove}
	    	
	    	\begin{proof}
	    		
	    		Let $S_n$ and $T_n$ denote the partial sums of $\sum a_n$ and $\sum b_n$ respectively.
	    		
	    		Because $\sum a_n$ converges, $S_n$ also converges and $S_n$ is bounded. Because $\sum b_n$ diverges, $T_n$ also diverges. 
	    		
	    		For sequences, we have already proven that---given $S_n$ bounded and $T_n$ divergent--- $S_n + T_n$ is also divergent. 
	    		
	    		But this means, that $\sum (a_n + b_n) = S_n + T_n$ diverges. 
	    	\end{proof}
	    	
	    	\part
	    	If $\sum a_n$ and $\sum b_n$ are divergent, then so is $\sum (a_n + b_n)$.
	    	
	    	We disprove this statement by using a counter example. Let $a_n = \frac{1}{n}$ and $b_n = -\frac{1}{n}$. Clearly $\sum a_n$ and $\sum b_n$ are divergent. Now:
	    	\[
		    	\sum (a_n + b_n) = \sum \left(\frac{1}{n} - \frac{1}{n}\right) = \sum 0 = 0
	    	\]
	    	In this case, both $\sum a_n$ and $\sum b_n$ are divergent, but $\sum (a_n + b_n)$ is convergent. 
	    \end{parts}
    
	    \question
	    Let $\sum_{k=1}^{\infty} a_k$ be an absolutely converging series.
	    
	    \begin{toprove}
	    	\[
		    	\left| \sum_{k=1}^{\infty} a_k \right| \leq \sum_{k=1}^{\infty} |a_k|
	    	\]
	    \end{toprove}
    
	    \begin{proof}
	    	Let $n \in \naturals$, then using the triangle inequality we have
	    	\[
		    	\left| \sum_{k=1}^{n} a_k \right| = |a_1 + a_2 + \cdots + a_n | \leq |a_1| + |a_2| + \cdots + |a_n| = \sum_{k=1}^{n} |a_k|\,.
	    	\]
	    	
	    	We also have
	    	\[
		    	\sum_{k=1}^{\infty} |a_k| = \sum_{k=1}^{n} |a_k| + \sum_{k=n+1}^{\infty} |a_k|\,,
	    	\]
	    	which, using what we found above and the fact that $|a_k| \geq 0$ for all $k$, yields
	    	\[
		    	\left| \sum_{k=1}^{n} a_k \right| \leq \sum_{k=1}^{\infty} |a_k|\,.
	    	\]
	    	Now as the sequence is absolutely convergent (i.e. $\sum_{k=1}^{\infty} |a_k| = Q \in \reals$), we know that each of the partial sums $\left|\sum_{k=1}^{n} a_k\right|$ is at most $Q$ and the limit of this sequence is bounded by this same value, $Q$.
	    	
	    	Simply put, we have
	    	\[
	    	\left| \sum_{k=1}^{\infty} a_k \right| = \left| \lim_{n\to\infty} \sum_{k=1}^{n} a_k \right| = \lim_{n\to\infty} \left|\sum_{k=1}^{n} a_k\right| \leq \lim_{n\to\infty} \sum_{k=1}^{n} |a_k| = \sum_{k=1}^{\infty} |a_k|\,.
	    	\]
	    \end{proof}
    
	    \question
	    Let $q \in \reals$, $|q| < 1$. Determine the values of the following series;
	    
	    \begin{parts}
	    	\part Define $a_k = \frac{q^k}{2}$. We are interested in 
	    	\[
	    	\sum_{k=7}^{\infty} a_k = \sum_{k=7}^{\infty} \frac{q^k}{2}\,.
	    	\]
	    	
	    	We have
	    	\begin{align*}
	    		\sum_{k=7}^{\infty} \frac{q^k}{2} &= \sum_{k=0}^{\infty} \frac{q^k}{2} - \sum_{k=0}^{6} \frac{q^k}{2} \\
	    		&= \frac{1}{2} \cdot \sum_{k=0}^{\infty} \frac{q^k}{2} - \sum_{k=0}^{6} \frac{q^k}{2} \\
	    		&= \frac{1}{2} \cdot \frac{1}{1-q} - \frac{1}{2} \cdot \frac{1 - q^7}{1-q} \\
	    		&= \frac{q^7}{2-2q}
	    	\end{align*}
	    	
	    	\part Define $a_k = \left(\frac{q}{2}\right)^k$. We are interested in 
	    	\[
	    	\sum_{k=7}^{\infty} a_k = \sum_{k=7}^{\infty} \left(\frac{q}{2}\right)^k\,.
	    	\]
	    	
	    	We have 
	    	\begin{align*}
	    		\sum_{k=7}^{\infty} \left(\frac{q}{2} \right)^k &= \sum_{k=0}^{\infty} \left(\frac{q}{2} \right)^k - \sum_{k=0}^{6} \left(\frac{q}{2} \right)^k \\
	    		&= \frac{1}{1 - \frac{q}{2}} - \frac{1 - \left(\frac{q}{2} \right)^7}{1 - \frac{q}{2}} \\
	    		&= \frac{q^7}{2^7 - 2^6 q}
	    	\end{align*}
	    	
	    	\part Define $a_k = \frac{2^{k/2}}{2^{2k}}$. We are interested in 
	    	\[
	    	\sum_{k=0}^{\infty} a_k = \sum_{k=0}^{\infty} \frac{2^{k/2}}{2^{2k}}\,.
	    	\]
	    	
	    	We have
	    	\begin{align*}
	    		\sum_{k=0}^{\infty} \frac{2^{\frac{k}{2}}}{2^{2k}} &= \sum_{k=0}^{\infty} \left(\frac{1}{2\sqrt{2}} \right)^k \\
	    		&= \frac{1}{1 - \frac{1}{2\sqrt{2}}} \\
	    		&= \frac{2}{7}\left(4 + \sqrt 2 \right)\,.
	    	\end{align*}
	    	
	    	\part Define $a_k = \frac{4^{k-5}}{5^{2k+2}}$. We are interested in 
	    	\[
	    	\sum_{k=0}^{\infty} a_k = \sum_{k=0}^{\infty} \frac{4^{k-5}}{5^{2k+2}}\,.
	    	\]
	    	
	    	We have
	    	\begin{align*}
	    		\sum_{k=0}^{\infty} \frac{4^{k - 5}}{5^{2k + 2}} &= \sum_{k=0}^{\infty} \frac{4^{-5}}{5^2} \cdot \sum_{k=0}^{\infty} \left(\frac{4}{25} \right)^k \\
	    		&= \frac{1}{4^5 \cdot 5^2} \cdot \frac{1}{1 - \frac{4}{25}} \\
	    		&= \frac{1}{25600} \cdot \frac{25}{25 - 4} \\
	    		&= \frac{1}{21504}\,.
	    	\end{align*}
	    \end{parts}
     \end{questions}
\end{document}