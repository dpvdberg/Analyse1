% !TeX spellcheck = en_US
\documentclass[week=11]{homework}
\usepackage{scrextend}
\date{\today}


\begin{document}
    \maketitle
    \thispagestyle{empty}
    \newpage
    \begin{questions}
		\let\firstquestion\question
		\renewcommand*{\question}{\vspace{7mm}\firstquestion}
        \firstquestion
		Let $\{b_n \}$ be a sequence of numbers. 
		\begin{inlinetoprove}
			If $\{b_n \}$ converges, then 
			\[
				\limsup b_n = \lim_{n \to \infty} b_n.
			\]
		\end{inlinetoprove}
		
		\question
		Let $\{b_n \}$ and $\{c_n \}$ be a sequences of numbers with $c_n \to 1$. 
		\begin{inlinetoprove}
			\[
				\limsup (c_n b_n) = \limsup b_n, \quad \limsup (b_n^{c_n}) = \limsup b_n
			\]
		\end{inlinetoprove}
		
		\question
		Let R be the radius of convergence for $\sum_{k=0}^{\infty} a_k (x - a)^k$ and suppose that $\lim_{n \to \infty} \left| \frac{a_{n+1}}{a_n} \right| = L$. Then:
		\begin{parts}
			\part if $L$ is a nonzero finite real number, $R = \frac{1}{L}$
			\part if $L=0$, $R = \infty$
			\part if $L = \infty$, $R = 0$. 
		\end{parts}
		
		\question
		Find both the radius and interval of convergence for the given series. Also, determine the values of x for which the series converges absolutely and those for which the series converges conditionally. 
		\begin{parts}
			\part $\sum_{k=1}^{\infty} k(x-2)^k$
			
			We can calculate:
			\[
				\limsup_{k \to \infty} \sqrt[k]{|k|} = 1			
			\]
			Now, we know that the radius of convergence is: $R = 1$. This means that the sum is convergent on the interval $x \in (1,3)$.  Now we examine the boundary points of this interval. 
			
			If $x = 1$ the sum is of the form $\sum_{k=1}^{\infty} (-1)^k k$. This clearly converges.  
			
			If $x = 3$ the sum is of the form $\sum_{k=1}^{\infty} k$. This also converges.  
			
			\part $\sum_{k=1}^{\infty} \frac{1}{k} \left( \frac{x}{2} \right)^k$
			
			First, we let $y = \frac{x}{2}$. Then, the sum is of the form $\sum_{k=1}^{\infty} \frac{1}{k} \left( y \right)^k$. 
			
			Now we can calculate:
			\[
				\limsup_{k \to \infty} \sqrt[k]{\frac{1}{k}} = 1
			\]
			Now, we know that the radius of convergence is: $R = 1$. This means that the sum is convergent on the interval $y \in (-1,1)$.  This also means the original sum is convergent on the interval $x \in (-2,2)$. Now we examine the boundary points of this interval. 
			
			If $x = -2$ the sum is of the form $\sum_{=1}^{\infty} (-1)^k \frac{1}{k}$. This is conditionally convergent by the theorem of Leibniz.  
			
			If $x = 2$ the sum is of the form $\sum_{=1}^{\infty} \frac{1}{k}$. This is divergent. 
			
			\part $\sum_{k=0}^{\infty} k \left( -\frac{1}{3} \right)^k (x-2)^k $
			
			We can calculate:
			\[
				\limsup_{k \to \infty} \sqrt[k]{|k \cdot \left( - \frac{1}{3} \right)^k |} = \limsup_{k \to \infty} 1 \cdot \frac{1}{3} = \frac{1}{3}
			\]
			Now, we know that the radius of convergence is: $R = 3$. This means that the sum is convergent on the interval $x \in (-1,5)$.  Now we examine the boundary points of this interval. 
			
			If $x = -1$ the sum is of the form $\sum_{k=0}^{\infty} k$. This is divergent.  
			
			If $x = 5$ the sum is of the form $\sum_{k=0}^{\infty} (-1)^k k$. This is divergent.  
			
			\part $\sum_{k=1}^{\infty} \frac{2^k k!}{k^k} x^k$
			
			We can calculate:
			\begin{align*}
			\limsup_{k \to \infty} \left| \frac{\frac{2^{k+1}(+1)!}{(k+1^{k+1})}}{\frac{2^k k!}{k^k}} \right| &= \limsup_{k \to \infty} 2(k+1) \cdot \left( \frac{k}{k+1} \right)^k \cdot \frac{1}{k+1} \\
			&= \limsup_{k \to \infty} 2 \cdot \left( \frac{k}{k+1} \right)^k \\
			&= \frac{2}{e}
			\end{align*}
			Now, we know that the radius of convergence is: $R = \frac{e}{2}$. This means that the sum is convergent on the interval $x \in (-\frac{e}{2}, \frac{e}{2})$.  Now we examine the boundary points of this interval. 
			
			If $x = - \frac{e}{2}$ the sum is of the form $\sum_{k=1}^{\infty} (-1)^k e^k \frac{k!}{k^k}$. This... I DONT KNOW HELP 
			
			If $x = \frac{e}{2}$ the sum is of the form $\sum_{k=1}^{\infty} e^k \frac{k!}{k^k}$. AGAIN, HELP.
			
			\part $\sum_{k=0}^{\infty} \frac{k}{3^{2k-1}} (x-1)^{2k}$
			
			First, we define $y = (x-1)^2$. Now, the sum is of the form: $\sum_{k=0}^{\infty} \frac{k}{3^{2k-1}} y^{k}$.
			We can calculate:
			\[
				\limsup_{k \to \infty} \sqrt[k]{\frac{k}{3^{2k-1}}} = \limsup_{k \to \infty} \frac{\sqrt[k]{k}}{3^2} = \frac{1}{9}
			\]
			Now, we know that the radius of convergence is: $R = 9$. This means that the sum is convergent on the interval $y \in (-9,9)$. This means that the original sum is convergent on the interval $x \in (-2,4)$. Now we examine the boundary points of this interval. 
			
			If $x = -2$ the sum is of the form $\sum_{k=0}^{\infty} - \frac{k}{3^{-1}}$. This is divergent.  
			
			If $x = 4$ the sum is of the form $\sum_{k=0}^{\infty} \frac{k}{3^{-1}}$. This is divergent.  
			
		\end{parts}
     \end{questions}
\end{document}