% !TeX spellcheck = en_US
\documentclass[week=11]{homework}
\usepackage{scrextend}
\date{\today}


\begin{document}
    \maketitle
    \thispagestyle{empty}
    \newpage
    \begin{questions}
		\let\firstquestion\question
		\renewcommand*{\question}{\vspace{7mm}\firstquestion}
        \firstquestion
		Let $\{b_n \}$ be a sequence of numbers. 
		\begin{toprove}
			If $\{b_n \}$ converges, then 
			\[
				\limsup b_n = \lim_{n \to \infty} b_n.
			\]
		\end{toprove}
		\begin{proof}
			First of all, we know that $b_n$ converges. Therefore:
			\[
				\exists_{b^*}: b_n \to b^*
			\]
			Now, because $b_n$ converges, this sequence only has one accumulation point, namely $b^*$. Therefore, the set of all accumulation points of $b_n$, which we call V, only consists of one point: 
			\[
				V = \{b^*\}
			\]
			Now, we can find $\limsup b_n$:
			\[
				\limsup b_n = \sup V = b^*
			\]
			We also know that:
			\[
				\lim_{n \to \infty} b_n = b^*
			\]
			Now, we have proven that:
			\[
				\limsup b_n = \lim_{n \to \infty} b_n
			\]
		\end{proof}
		\question
		Let $\{b_n \}$ and $\{c_n \}$ be a sequences of numbers with $c_n \to 1$. 
		\begin{toprove}
			\[
				\limsup (c_n b_n) = \limsup b_n, \quad \limsup (b_n^{c_n}) = \limsup b_n\,.
			\]
		\end{toprove}
		\begin{proof}
			We have proven that (college 4, question 3) if a sequence is divergent, we can find a subsequence that also diverges to the same limit. Also, if a sequence has a accumulation point, then we can also find a subsequence that converges to this same point.
			
			By definition of limes superior;
			\[
				\limsup b_n = 
				\begin{cases}
					+\infty & b_n \text{ unbounded above,} \\
					\sup V & V \neq \emptyset \text{ and } b_n \text{ bounded above,} \\
					-\infty & V = \emptyset \text{ and } b_n \text{ bounded above.} \\
				\end{cases}
			\]
			Using what we stated above, in any case we can find a subsequence $b_{n_k}$ such that
			\begin{equation} \label{subsequence_to_limsup} \tag{$\star$}
				b_{n_k} \xrightarrow{k\to\infty} \limsup b_n\,.
			\end{equation}
			
			Note that $\sup V$ might not be an element of $V$, but we can still find a subsequence that converges to $\limsup b_n$ in this case, as proven in the lectures.
			
			We shall now prove $\limsup (c_n b_n) = \limsup b_n$.
			\begin{parts}
				\part $\limsup b_n = \infty$.
				
				In this case $b_n$ is unbounded from above. We know $c_n$ is convergent, whence it is also bounded. Therefore $c_n b_n$ is unbounded from above and by definition $\limsup (c_n b_n) = \infty$.
				
				\part $\limsup b_n = L \in \reals$.
				
				Now $L = \sup V$, where $V$ is the supremum over the set of accumulation points of $b_n$. Using \ref{subsequence_to_limsup}, we can find a subsequence $b_{n_k}$ that converges to $L$. As $c_n$ is convergent, every subsequence converges to the same limit, hence $c_{n_k} \xrightarrow{k \to \infty} 1$ and $c_{n_k} b_{n_k} \xrightarrow{k \to \infty} L$. Note that this must be the $\limsup$ of $c_n b_n$, for we know $c_n b_n$ is bounded above, hence $\limsup c_n b_n \in (-\infty, \sup V']$, where $V'$ is the set of accumulation points of $c_n b_n$. Now suppose that $\sup V' = L' > L$, then using \ref{subsequence_to_limsup} we can find a index sequence $(n_l)$ such that $c_{n_l} b_{n_l} \xrightarrow{l \to \infty} L'$. As $c_{n_l} \xrightarrow{l \to\infty} 1$ we have $b_{n_l} \xrightarrow{l \to \infty} L'$. But then $L = \sup V < L' \in V \quad \lightning$.
				
				In conclusion, in this case $\limsup c_n b_n = \limsup b_n = L$.
				
				\part $\limsup b_n = -\infty$.
				
				In this case $V = \emptyset$ and $b_n$ is bounded above, i.e. $b_n$ is unbounded from below and $b_n \to -\infty$. Now suppose that $V'$ of $c_n b_n$ is not empty. Let $q \in V'$. Then there exists a subsequence $c_{n_k} b_{n_k} \xrightarrow{k \to \infty} q$. Now $c_{n_k} \xrightarrow{k\to\infty} 1$, hence $b_{n_k} \xrightarrow{k \to \infty} q$, which is not possible, as $b_n$ goes to minus infinity $\lightning$. Hence $V' = \emptyset$ and since $b_n$ and $c_n$ are bounded above, so is $c_n b_n$ and therefore $\limsup (c_n b_n) = -\infty$.
			\end{parts}
		\end{proof}
		
		\question
		Let R be the radius of convergence for $\sum_{k=0}^{\infty} a_k (x - a)^k$ and suppose that $\lim_{n \to \infty} \left| \frac{a_{n+1}}{a_n} \right| = L$. Then:
		\begin{parts}
			\part if $L$ is a nonzero finite real number, $R = \frac{1}{L}$. 
			\begin{proof}
			Let $z_k := a_k (x-a)^k$. Then
			\[
				\left| \frac{z_{k+1}}{z_k} \right| = \left| \frac{a_{k+1}}{a_k} \right| |x-a| \xrightarrow{k \to \infty} L \cdot |x-a|\,.
			\]
			In this case, by Cauchy's root test, we know that $\sum z_k$ is convergent if $L \cdot |x-a| < 1 \implies |x-a| < 1/L$ and divergent if $L \cdot |x-a| > 1 \implies |x-a| > 1/L$. Hence, the radius of convergence is $1/L$.
			\end{proof}
	
			\part if $L=0$, $R = \infty$.
			
			\begin{proof}	
			Let $z_k := a_k (x-a)^k$. Then
			\[
				\left| \frac{z_{k+1}}{z_k} \right| = \left| \frac{a_{k+1}}{a_k} \right| |x-a| \xrightarrow{k \to \infty} 0 \cdot |x-a| = 0 < 1\,.
			\]
			
			By Cauchy's root test, $\sum z_k$ is convergent for all $x$, hence the radius of convergence is infinity.
			\end{proof}	
			
			\part if $L = \infty$, $R = 0$. 
			\begin{proof}
		
			Let $z_k := a_k (x-a)^k$. Then for $x \neq a$
			\[
				\left| \frac{z_{k+1}}{z_k} \right| = \left| \frac{a_{k+1}}{a_k} \right| |x-a| \xrightarrow{k \to \infty} \infty\,.
			\]
			By Cauchy's root test, $\sum z_k$ is divergent for all $x \neq a$, hence the radius of convergence is zero.
			\end{proof}
		\end{parts}
		
		\question
		Find both the radius and interval of convergence for the given series. Also, determine the values of x for which the series converges absolutely and those for which the series converges conditionally. 
		\begin{parts}
			\part $\sum_{k=1}^{\infty} k(x-2)^k$
			\part $\sum_{k=1}^{\infty} \frac{1}{k} \left( \frac{x}{2} \right)^k$
			\part $\sum_{k=0}^{\infty} k \left( -\frac{1}{3} \right)^k (x-2)^k $
			\part $\sum_{k=1}^{\infty} \frac{2^k k!}{k^k} x^k$
			\part $\sum_{k=0}^{\infty} \frac{k}{3^{2k-1}} (x-1)^{2k}$
		\end{parts}
     \end{questions}
\end{document}