% !TeX spellcheck = en_US
\documentclass[week=6]{homework}
\usepackage{scrextend}
\date{\today}


\begin{document}
    \maketitle
    \thispagestyle{empty}
    \newpage
    \begin{questions}
		\let\firstquestion\question
		\renewcommand*{\question}{\vspace{7mm}\firstquestion}
        \firstquestion
        % INSERT Q1
        
        \question
        % NAH..
        
        \question
        % INSERT Q3
        
        \question
        Determine whether or not the given series converges uniformly on the indicated interval.
        \begin{parts}
        	\part $\displaystyle \sum \frac{x^k}{k^2}$ with $x \in [0,1]$.
        	
        	\part $\displaystyle \sum x^ke^{-kx}$ with $x \in [0,\infty)$.
        	
        	\part $\displaystyle \sum k^re^{-kx}$ with $x \in [a,\infty)$, where $a > 0$ and $r \in \reals$ a constant.
        	
        	\part $\displaystyle \sum \frac{1}{x^k + 1}$ with $x \in (0,1]$.
        	
        	\part $\displaystyle \sum \frac{1}{x^k + 1}$ with $x \in (1,\infty)$.
        \end{parts}
	    
	    \question
	    % INSERT Q5
	    
	    \question
	    Let $[a,b]$ be a bounded and closed interval. Let $(f_n)$ be a sequence of non-negative continuous functions on $[a,b]$ such that $\sum f_n$ converges pointwise to a limiting continuous function $s$, given by
	    \[
		    s(x) = \sum_n f_n(x)\,.
	    \]
	    \begin{toprove}
	    	The function series $\sum f_n$ is uniformly convergent.
	    \end{toprove}
	    \begin{proof}
	    	Let $s_n$ denote the partial sums of $s$, i.e.
	    	\[
	    	s_n(x) = \sum_{k=1}^{n} f_k(x)\,.
	    	\]
	    	To apply Dini's theorem, we need
	    	\begin{enumerate}
	    		\item \label{Q6:interval} $s_n$ and $s$ are defined on a closed interval.
	    		\item \label{Q6:pointwise} $s_n$ converges pointwise to $s$
	    		\item \label{Q6:continuous} $s_n$ is continuous for all $n$, as well is the limiting function $s$.
	    		\item \label{Q6:increasing} $s_n$ is an increasing sequence.
	    	\end{enumerate}
	    	We have that \ref{Q6:interval} and \ref{Q6:pointwise} are satisfied by the antecedent of what is to be proven.
	    	
	    	As $f_n$ is continuous for all $n$, by definition of $s_n$ and the fact that summation maintains continuity, we find that $s_n$ is continuous. Note that we already have continuity of $s$. Hence, \ref{Q6:continuous} is satisfied.
	    	
	    	As $f_n$ is non-negative, for all $n$ and $x \in [a,b]$ we have $f_n(x) \geq 0$. But then for all $n$ and $x$ we find
	    	\begin{align*}
	    	s_n(x) &= \sum_{k=1}^{n} f_n(x) \\
	    	&\leq f_{n+1}(x) + \sum_{k=1}^{n} f_n(x) \\
	    	&= \sum_{k=1}^{n+1} f_n(x) = s_{n+1}(x)\,.
	    	\end{align*}
	    	Which means \ref{Q6:increasing} is satisfied as well and by Dini's theorem we find that $s_n$ converges uniformly to $s$.
	    \end{proof}
     \end{questions}
\end{document}