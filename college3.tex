% !TeX spellcheck = en_US
\documentclass[week=3]{homework}

\date{\today}

\begin{document}
    \maketitle
    \thispagestyle{empty}
    \newpage
    \begin{questions}
		\let\firstquestion\question
		\renewcommand*{\question}{\vspace{7mm}\firstquestion}
        \firstquestion
        Let $\{a_n\}$, $\{b_n\}$ be sequences with limits $a_n \to a^*$ and $b_n \to b^*$, where $a^*,b^* \in \reals$. Let $c\in\reals$ be arbitrary.
        
        Prove the following statements:
        \begin{parts}
        	\part
        	\begin{inlinetoprove}
        		$ca_n \to ca^*$
        	\end{inlinetoprove}
	        \begin{proof}
	        	% INSERT Q1a
	        \end{proof}
	        
	        \part
	        \begin{inlinetoprove}
	        	$a_n + b_n \to a^* + b^*$
	        \end{inlinetoprove}
	        \begin{proof}
	        	% INSERT Q1b
	        \end{proof}
        \end{parts}
    
	    \question
	    Let $\{a_n\}$ be a sequence with improper limit $a_n \to \pm\infty$ and $a_n \neq 0$ for all $n$. Let $\{b_n\}$ be a bounded sequence.
	    
	    Show the following statements:
	    \begin{parts}
	    	\part
	    	\begin{inlinetoprove}
	    		$a_n + b_n \to \pm \infty$
	    	\end{inlinetoprove}
	    	\begin{proof}
	    		There are two possibilities:
	    		\begin{itemize}
		    		\item $a_n \to \infty$		    		
		    	\end{itemize}
		    	
		    	We know that:
		    	\[
			    	\forall_{M \in \reals}\exists_{n_0 \in \naturals} \forall_{n \ge n_0}: a_n > M
		    	\]
		    	We also know that $b_n$ is bounded, so:
		    	\[
			    	\exists_{K \in \reals} \forall_{n \in \naturals}: b_n > K
		    	\]
		    	
		    	Let $M \in \reals$ be given and let $S = M + K$. For $n \ge n_0$ it now holds that:
		    	\[
			    	a_n + b_n > a_n + K > M + K = S
		    	\]
		    	Thus: 
		    	\[
			    	\forall_{S \in \reals}\exists_{n_0 \in \naturals} \forall_{n \ge n_0}: a_n + b_n > S
		    	\]
		    	So:
		    	\[
			    	a_n + b_n \to \infty
		    	\]
		    	
		    	\begin{itemize}
		    		\item $a_n \to - \infty$			    
	    		\end{itemize}
	    		
	    		In this case, we know:
	    		\[
		    		\forall_{N \in \reals}\exists_{n_1 \in \naturals} \forall_{n \ge n_1}: a_n < N
	    		\]
	    		Again, $b_n$ is bounded, so:
	    		\[
		    		\exists_{L \in \reals} \forall_{ n \in \naturals}: b_n < L
	    		\]
	    		Let $N \in \reals$ be arbitrary and let $T = N + L$. For $n \ge n_1$ it holds that:
	    		\[
		    		a_n + b_n < N + b_n < N + L = T
	    		\]
	    		Thus:
	    		\[
		    		\forall_{T \in \reals}\exists_{n_1 \in \naturals} \forall_{n \ge n_1}: a_n + b_n < T
	    		\]
	    		Thus: 
	    		\[
		    		a_n + b_n \to - \infty
	    		\]
	    		
	    		We have now proven that, if $a_n \to \pm\infty$, $a_n \neq 0$ for all $n$ and $b_n$ is bounded, then $a_n + b_n \to \pm \infty$.
	    	\end{proof}
    	
	    	\part
	    	\begin{inlinetoprove}
	    		$\frac{b_n}{a_n} \to 0$
	    	\end{inlinetoprove}
	    	\begin{proof}
	    		Again, we distinguish this proof in two different parts:
	    		\begin{itemize}
	    			\item $a_n \to \infty$
	    		\end{itemize}
	    		We know that $b_n$ is bounded, so:
	    		\[
		    		\exists_{K > 0 \in \reals} \forall_{n \in \naturals}: b_n < K
	    		\]
	    		Also, since $a_n \to \infty$:
	    		\[
		    		\forall_{\epsilon \in \reals} \exists_{n_0 \in \naturals} \forall_{n \ge n_0}: a_n > \frac{K}{\epsilon}
	    		\]
	    		But then it also holds that:
	    		\[
		    		\forall_{\epsilon > 0 \in \reals} \exists_{n_0 \in \naturals} \forall_{n \ge n_0}: \frac{1}{a_n} < \frac{\epsilon}{K}
	    		\]
	    		Let $\epsilon > 0$ be given. For $n \ge n_0$ it holds that:
	    		\[
		    		\left| \frac{b_n}{a_n} \right| < \left| \frac{K}{a_n} \right| < \left| K \cdot \frac{\epsilon}{K} \right| = \epsilon
	    		\]
	    		So:
	    		\[
		    		\forall_{\epsilon > 0} \exists_{n_0 \in \naturals} \forall_{n \ge n_0}: \left| \frac{b_n}{a_n} - 0\right| < \epsilon
	    		\]
	    		Thus, in this case $\frac{b_n}{a_n} \to 0$.
	    		
	    		\begin{itemize}
	    			\item $a_n \to - \infty$
	    		\end{itemize}
	    		We know that $b_n$ is bounded, so:
	    		\[
	    		\exists_{K > 0 \in \reals} \forall_{n \in \naturals}: b_n < K
	    		\]
	    		Also, since $a_n \to -\infty$:
	    		\[
	    		\forall_{\epsilon \in \reals} \exists_{n_1 \in \naturals} \forall_{n \ge n_1}: a_n < \frac{K}{\epsilon}
	    		\]
	    		So, then it holds that:
	    		\[
	    		\forall_{\epsilon > 0 \in \reals} \exists_{n_1 \in \naturals} \forall_{n \ge n_1}: a_n < -\frac{K}{\epsilon}
	    		\]
	    		But then:
	    		\[
	    		\forall_{\epsilon > 0 \in \reals} \exists_{n_1 \in \naturals} \forall_{n \ge n_1}: \frac{1}{a_n} < -\frac{\epsilon}{K}
	    		\]
	    		Let $\epsilon > 0$ be given. For $n \ge n_0$ it holds that:
	    		\[
	    		\left| \frac{b_n}{a_n} \right| < \left| \frac{K}{a_n} \right| < \left| K \cdot -\frac{\epsilon}{K} \right| = \epsilon
	    		\]
	    		So:
	    		\[
	    		\forall_{\epsilon > 0} \exists_{n_1 \in \naturals} \forall_{n \ge n_1}: \left| \frac{b_n}{a_n} - 0\right| < \epsilon
	    		\]
	    		Thus, also in this case $\frac{b_n}{a_n} \to 0$.
	    		
	    		We have now proven that in both cases $\frac{b_n}{a_n} \to 0$.
	    	\end{proof}
	    \end{parts}
    
	    \question
	    Determine if the following sequences converge en find their (possibly improper) limits if such a limit exists. Show what rules and theorems are used.
	    
	    \begin{alignat*}{3}
	    a_n &:= \frac{1}{n^2} - \sqrt n\,,  \qquad &&b_n := \frac{1}{\sqrt n } + (-1)^nn\,,\qquad &&c_n := \left(-\frac{3}{n}\right)^n\,, \\
	    d_n &:= \frac{(1.01)^n}{n^{2007}}\,, \qquad &&e_n := \frac{n^2-5n+4}{n^3+8n^2+2}\,, \qquad &&f_n := \sqrt[n]{n^3}\,, \\
	    g_n &:= \frac{3^n - n^3}{2^n + n^2 + 7}\,, \qquad &&h_n := \frac{\sqrt n - \frac{1}{2n}}{\sqrt[3]{n} + \frac{1}{n^2}}\,, \qquad &&k_n := \frac{n^8 + n^5}{4n^8 + 5n^6 + 3}\,.
	    \end{alignat*}
	    \begin{parts}
	    	\part
	    	% INSERT Q3a
	    	
	    	\part
	    	% INSERT Q3b
	    	
	    	% ETC..
	    \end{parts}
    
	    \question Let the sequence $\{a_n\}$ be given by
	    \[
		    a_0 = 1\,, \qquad a_{n+1} = a_n + \frac{1}{a_n}\,, \qquad n = 0,1,2,\ldots
	    \]
	    
	    Determine whether $\{a_n\}$ converges and find its limit if it exists.
	    
	    % INSERT Q4
     \end{questions}
\end{document}