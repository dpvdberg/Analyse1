% !TeX spellcheck = en_US
\documentclass[week=4]{homework}

\date{\today}

\begin{document}
    \maketitle
    \thispagestyle{empty}
    \newpage
    \begin{questions}
		\let\firstquestion\question
		\renewcommand*{\question}{\vspace{7mm}\firstquestion}
        \firstquestion
        Determine the limits of the following sequences:
        
        \[
	        a_n := \left(1 + \frac{1}{n} \right)^{2n}, \quad b_n := \left(\frac{n+8}{n+7} \right)^{n+4}, \quad c_n := \left(n + \frac{1}{n^2} \right)^{n}, \quad d_n := \left(1 - \frac{1}{n} \right)^n
        \]
        
        \question 
        Show: Each subsequence of a convergent sequence converges to the same limit. 
        
        \question 
        Show: Each sequence that is unbounded above / below has a subsequence that goes to $+ \infty / - \infty$. Conclude: Every sequence of real numbers has a finite or improper convergent subsequence. 
        
        \question
        Find all accumulation points of the following sequences:
        \[
	        a_n := \frac{k}{dn}, \quad b_n := \left(-1-\frac{1}{n}\right)^n
        \]
        
        (Show that you have indeed found all accumulation points!)
        
        \question
        For a sequence $\{a_n \} $, define $V(\{a_n \})$ the set of accumulation points. Give examples of sequences $\{a_n \}$ such that:
        
        \begin{parts}
        	\part $V(\{a_n \})$ has exactly 4 elements 
        	
        	Define $a_n = n \pmod 4 = \{0,1,2,3,\enskip0,1,2,3,\enskip0,1,2,3,\ldots\}$, then $V(\{a_n\}) = \{0,1,2,3\}$.
        	\part $V(\{a_n \})$ has infinitely many elements
        	
        	Define
        	\[
	        	a_n =  n + 1 - \frac{\ceil{p_n}(\ceil{p_n}-1)}{2} = \{0,1,\enskip0,1,2,\enskip0,1,2,3,\ldots\}\,,
	        \]
	        where
	        \[
		        p_n := \frac{\sqrt{8n+9}-1}{2}\,,
		    \]
		    then $V(\{a_n\}) = \{0,1,2,3,\ldots\} = \naturals$.
        	\part $\star$ $V(\{a_n \}) = [0, 1]$
        \end{parts}
     \end{questions}
\end{document}