% !TeX spellcheck = en_US
\documentclass[week=4]{homework}

\date{\today}

\begin{document}
    \maketitle
    \thispagestyle{empty}
    \newpage
    \begin{questions}
		\let\firstquestion\question
		\renewcommand*{\question}{\vspace{7mm}\firstquestion}
        \firstquestion
        Determine the limits of the following sequences:
        
        \[
	        a_n := \left(1 + \frac{1}{n} \right)^{2n}, \quad b_n := \left(\frac{n+8}{n+7} \right)^{n+4}, \quad c_n := \left(n + \frac{1}{n^2} \right)^{n}, \quad d_n := \left(1 - \frac{1}{n} \right)^n
        \]
        
        \question 
        Show: Each subsequence of a convergent sequence converges to the same limit. 
        
        \question 
        Show: Each sequence that is unbounded above / below has a subsequence that goes to $+ \infty / - \infty$. Conclude: Every sequence of real numbers has a finite or improper convergent subsequence. 
        
        \begin{inlinetoprove}
        	Each sequence that is unbounded above has a subsequence that goes to $\infty$.
        \end{inlinetoprove}
        \begin{proof}
        	Let $a_n$ be a random sequence unbounded above. Then it holds that:
        	\[
	        	\forall_{\epsilon > 0} \exists_{n \in \naturals} : a_n > \epsilon
        	\]
	        So, we also know that:
	        \[
		        \forall_{\epsilon > 0} \exists_{n_k \in \naturals} : a_{n_{k}} > \epsilon
	        \]
	        Now, let $a_{n_{k-1}} = M$ and $\epsilon = M$. Then: 
	        \[
		        \exists_{n_k \in \naturals} : a_{n_{k}} > M
	        \]
	        Take $n_0 = 0$ as a starting point. Now $a_{n_k}$ is a strictly increasing (sub)sequence that is unbounded above. Therefore it holds that:
	        \[
		        \forall_{\epsilon > 0} \exists_{k^* \in \naturals} \forall_{k \ge k^*} : a_{n_k} > \epsilon
	        \]
	        So:
	        \[
		        a_{n_k} \to \infty 
	        \]
	        We have now shown that each sequence that is unbounded above has a subsequence that goes to $\infty$.
        \end{proof}
        
        \begin{inlinetoprove}
        	Each sequence that is unbounded below has a subsequence that goes to $- \infty$.
        \end{inlinetoprove}
        \begin{proof}
        	Let $a_n$ be a random sequence unbounded below. Then it holds that:
        	\[
        	\forall_{\epsilon < 0} \exists_{n \in \naturals} : a_n < \epsilon
        	\]
        	So, we also know that:
        	\[
        	\forall_{\epsilon < 0} \exists_{n_k \in \naturals} : a_{n_{k}} < \epsilon
        	\]
        	Now, let $a_{n_{k-1}} = M$ and $\epsilon = M$. Then: 
        	\[
        	\exists_{n_k \in \naturals} : a_{n_{k}} < M
        	\]
        	Take $n_0 = 0$ as a starting point. Now $a_{n_k}$ is a strictly decreasing (sub)sequence that is unbounded below. Therefore it holds that:
        	\[
        	\forall_{\epsilon > 0} \exists_{k^* \in \naturals} \forall_{k \ge k^*} : a_{n_k} > \epsilon
        	\]
        	So:
        	\[
        	a_{n_k} \to -\infty 
        	\]
        	We have now shown that each sequence that is unbounded below has a subsequence that goes to $- \infty$.
        \end{proof}
        
        \begin{inlinetoprove}
        	Every sequence of real numbers has a finite or improper convergent subsequence. 
        \end{inlinetoprove}
        \begin{proof}
        	Every sequence $a_n$ falls under one of the following categories:
        	\begin{itemize}
        		\item $a_n$ bounded
        	In this case, accourding to the theorem of Bolzano-Weierstrass, the sequence $a_n$ has a convergent subsequence. It therefore has a finite convergent subsequence.  
        		\item $a_n$ is unbounded above
        	We have already proven that if $a_n$ unbounded above, then it has a subsequence that goes to $\infty$. It therefore has an improper convergent subsequence. 
        		\item $a_n$ is unbounded below
        	We have already proven that if $a_n$ unbounded below, then it has a subsequence that goes to $- \infty$. It therefore has an improper convergent subsequence. 
        		\item $a_n$ is unbounded
	        In this case, we can use the fact that we have proven that if $a_n$ is unbounded above / below, then it has a subsequence that goes to $\infty / - \infty$. This sequence therefore has an improper convergent subsequence. 
        	\end{itemize}
        	In conclusion: in all cases, $a_n$ has a finite or improper convergent subsequence. 
        \end{proof}
        \question
        Find all accumulation points of the following sequences:
        \[
	        a_n := \frac{k}{dn}, \quad b_n := \left(-1-\frac{1}{n}\right)^n
        \]
        
        (Show that you have indeed found all accumulation points!)
        
        \question
        For a sequence $\{a_n \} $, define $V(\{a_n \})$ the set of accumulation points. Give examples of sequences $\{a_n \}$ such that:
        
        \begin{parts}
        	\part $V(\{a_n \})$ has exactly 4 elements 
        	\part $V(\{a_n \})$ has infinitely many elements
        	\part $\star$ $V(\{a_n \}) = [0, 1]$
        \end{parts}
     \end{questions}
\end{document}