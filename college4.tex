% !TeX spellcheck = en_US
\documentclass[week=4]{homework}

\date{\today}

\begin{document}
    \maketitle
    \thispagestyle{empty}
    \newpage
    \begin{questions}
		\let\firstquestion\question
		\renewcommand*{\question}{\vspace{7mm}\firstquestion}
        \firstquestion
        Determine the limits of the following sequences:
        
        \[
	        a_n := \left(1 + \frac{1}{n} \right)^{2n}, \quad b_n := \left(\frac{n+8}{n+7} \right)^{n+4}, \quad c_n := \left(1 + \frac{1}{n^2} \right)^{n}, \quad d_n := \left(1 - \frac{1}{n} \right)^n
        \]
        
        By definition of $e$ we have
        \[
	        e_n := \left( 1 + \frac{1}{n} \right)^n \xrightarrow{n\to\infty} e\,.
        \]
        
        \begin{parts}
        	\part
        	We rewrite
        	\[
	        	a_n = \left(\left( 1 + \frac{1}{n} \right)^n\right)^2\,,
        	\]
        	and use the fact that for any convergent sequence $q_n \to q^*$ and $p \in \reals$ we have $(q_n)^p \to (q^*)^p$ to find
        	\[
	        	a_n = (e_n)^2 \xrightarrow{n\to\infty} e^2\,.
        	\]
        	
        	\part
        	We write
        	\[
	        	b_{n-7} = \left( \frac{n+1}{n} \right)^{n-3} = \left( 1 + \frac{1}{n} \right)^n \cdot \left( 1 + \frac{1}{n} \right)^{-3}\,.
        	\]
        	
        	We know
        	\begin{align*}
        		1 + \frac{1}{n} &\xrightarrow{n\to\infty} 1 \\
        		\therefore \left(1 + \frac{1}{n}\right)^{-3} &\xrightarrow{n\to\infty} 1^{-3} = 1\,.
        	\end{align*}
        	
        	For two convergent sequences $q_n \to q^*$ and $p_n \to p^*$ we have $\lim_{n\to\infty} q_n = \lim_{n\to\infty} q_{n-7} = q^*$ and $q_np_n \to q^*p^*$, such that
        	\[
	        	\lim_{n\to\infty} b_n = \lim_{n\to\infty} b_{n-7} = \lim_{n\to\infty} e_n \cdot \lim_{n\to\infty} \left(1 + \frac{1}{n}\right)^{-3} = e \cdot 1^{-3} = e\,.
        	\]
        	
        	\part
        	For any $n \neq 0$ we have $1/n^2 > 0$ such that
        	\[
	        	c_n = \left(1 + \frac{1}{n}\right)^n \geq 1^n = 1\,.
        	\]
        	
        	Also, we can write
        	\[
	        	c_n = \left(1 + \frac{1}{n^n}\right)^n = \left(1 + \frac{1}{n^2}\right)^{n^2 \cdot \frac{1}{n}} = \left(\left(1 + \frac{1}{n^2}\right)^{n^2}\right)^\frac{1}{n} = \sqrt[n]{\left(1 + \frac{1}{n^2}\right)^{n^2}}\,,
        	\]
        	which, using the fact that for any convergent sequence $q_n \to q^*$ we have $\sqrt[n]{q_n} \xrightarrow{n\to\infty} 1$, yields
        	\[
	        	c_n = \sqrt[n]{e_n} \xrightarrow{n\to\infty} 1\,.
        	\]
        	
        	\part
        	Using $u = -n$ we can write
        	\[
	        	d_n = \left(1 + \frac{1}{u}\right)^{-u} = \frac{1}{\left(1 + \frac{1}{u}\right)^u}\,.
        	\]
        	
        	For any convergent sequence $q_n \to q^*$, where $q_n \neq 0$ for all $n$ we have $1/q_n \xrightarrow{n\to\infty} 1/q^*$. Using this and the fact that $e_n \neq 0$ for all $n$ we find
        	\[
	        	d_n = \frac{1}{e_n} \xrightarrow{n\to\infty} \frac{1}{e}\,.
        	\]
        \end{parts}
        
        
        \question 
        \begin{inlinetoprove}
        	Each subsequence of a convergent sequence converges to the same limit. 
        \end{inlinetoprove}
        \begin{proof}
        	Let $a_n$ be a convergent sequence, $a_n \xrightarrow{n\to\infty} a^*$. Let $\{n_k\}$ be an index sequence of the subsequence $\{a_{n_k}\}$ and let $\epsilon > 0$ be arbitrary. We need to show
        	\[
	        	\exists_{k^* \in \naturals}\forall_{k \geq k^*}: |a_{n_k} - a^*| < \epsilon\,.
        	\]
        	
        	We have $a_n \xrightarrow{n\to\infty} a^*$, which means
        	\[
	        	\exists_{n_1 \in \naturals}\forall_{n \geq n_1}: |a_n - a^*| < \epsilon\,.
        	\]
        	By definition of a index sequence for a subsequence, we have that $n_{k+1} > n_k$ and $n_k \xrightarrow{k\to\infty} \infty$, therefore
        	\[
	        	\forall_{M > 0}\exists_{k_1 \in \naturals}\forall_{k \geq k_1}: n_k > M\,.
        	\]
        	
        	Choose $M = n_1$ and $k^* = k_1$, then for all $k \geq k^* = k_1$ we find $n_k > n_1$, which implies
        	\[
	        	|a_{n_k} - a^*| < \epsilon\,.
        	\]
        	
        	This in turn implies $a_{n_k} \xrightarrow{n\to\infty} a^*$.
        \end{proof}
        
        \question 
        Show that for each sequence that is unbounded above / below has a subsequence that goes to $+ \infty / - \infty$. Conclude: Every sequence of real numbers has a finite or improper convergent subsequence. 
        
        \begin{inlinetoprove}
        	Each sequence that is unbounded above has a subsequence that goes to infinity.
        \end{inlinetoprove}
        \begin{proof}
        	Let $a_n$ be a random sequence unbounded above. Then it holds that:
        	\[
	        	\forall_{M > 0} \exists_{n \in \naturals} : a_n > M\,.
        	\]
	        Let $n_k$ be an index sequence. As any such sequence is strictly increasing and $n_k \xrightarrow{k\to\infty} \infty$ we have
	        \[
		        \forall_{M > 0} \exists_{k \in \naturals} : a_{n_{k}} > M\,.
	        \]
	        Now, let $a_{n_{k-1}} = M$. Then
	        \[
		        \exists_{k \in \naturals} : a_{n_{k}} > M = a_{n_{k-1}}\,.
	        \]
	        Take $a_{n_0} = a_0$ as a starting point. Now $a_{n_k}$ is a strictly increasing (sub)sequence that is unbounded above. Therefore it holds that:
	        \[
		        \forall_{M > 0} \exists_{k^* \in \naturals} \forall_{k \ge k^*} : a_{n_k} > M \implies a_{n_k} \xrightarrow{k \to \infty} \infty\,.
	        \]
	        We have now shown that each sequence that is unbounded above has a subsequence that goes to infinity.
        \end{proof}
        
        \begin{inlinetoprove}
        	Each sequence that is unbounded below has a subsequence that goes to $- \infty$.
        \end{inlinetoprove}
        \begin{proof}
        	Let $a_n$ be a random sequence unbounded below. Then it holds that:
        	\[
        	\forall_{M < 0} \exists_{n \in \naturals} : a_n < M
        	\]
        	Let $n_k$ be an index sequence. As any such sequence is strictly increasing and $n_k \xrightarrow{k\to\infty} \infty$ we have
        	\[
        	\forall_{M < 0} \exists_{k \in \naturals} : a_{n_{k}} < M
        	\]
        	Now, let $a_{n_{k-1}} = M$. Then
        	\[
        	\exists_{k \in \naturals} : a_{n_{k}} < M = a_{n_{k-1}}
        	\]
        	Take $a_{n_0} = a_0$ as a starting point. Now $a_{n_k}$ is a strictly decreasing (sub)sequence that is unbounded below. Therefore it holds that:
        	\[
        	\forall_{M > 0} \exists_{k^* \in \naturals} \forall_{k \ge k^*} : a_{n_k} < M \implies a_{n_k} \xrightarrow{k \to \infty} -\infty\,.
        	\]
        	We have now shown that each sequence that is unbounded above has a subsequence that goes to minus infinity.
        \end{proof}
        
        \begin{inlinetoprove}
        	Every sequence of real numbers has a finite or improper convergent subsequence. 
        \end{inlinetoprove}
        \begin{proof}
        	Every sequence $a_n$ falls under one of the following categories:
        	\begin{itemize}
        		\item $a_n$ bounded.
        		
	        	In this case, according to Bolzano-Weierstrass' theorem, the sequence $a_n$ has a subsequence $a_{n_k}$ converging to some finite number.
        	
        		\item $a_n$ is unbounded (above, below or both).
        		
		        In this case, we can use the fact that we have proven that if $a_n$ is unbounded above / below, then it has a subsequence that goes to $\infty / - \infty$. This sequence therefore has an improper convergent subsequence. 
        	\end{itemize}
        	In conclusion: in all cases, $a_n$ has a subsequence that converges or has an improper limit. 
        \end{proof}
        \question
        Find all accumulation points of the following sequences:
        \[
	        a_n := \frac{(-2)^n - 2n}{2^{n+1}+n^2}, \quad b_n := \left(-1-\frac{1}{n}\right)^n
        \]
        
        (Show that you have indeed found all accumulation points!)
        
        \begin{parts}
        	\part \label{4:A} We look at the sequence $a_n$ for subsequences $a_{2n}$ and $a_{2n+1}$. We find
        	\begin{align*}
        		a_{2n} &= \frac{(-2)^{2n} - 4n}{2^{2n+1}+(2n)^2} \\
	        		&= \frac{4^n - 4n}{2\cdot 4^n + 4n^2} \\
	        		&= \frac{1 - \frac{4n}{4^n}}{2 + \frac{4n^2}{4^n}} \xrightarrow{n\to\infty} \frac{1}{2}\,, \\
	        		\\
		        a_{2n+1} &= \frac{(-2)^{2n+1} - 2(2n+1)}{2^{2n+2}+(2n+1)^2} \\
			        &= \frac{-2\cdot 4^n - 4n - 2}{4\cdot 4^n + 4n^2 + 4n + 1} \\
			        &= \frac{-2 - \frac{4n-2}{4^n}}{4 + \frac{4n^2+4n+1}{4^n}} \xrightarrow{n\to\infty} -\frac{1}{2}\,.
        	\end{align*}
        	
        	Now suppose we have some subsequence $a_{n_k}$ that is not $a_{2n}$ or $a_{2n+1}$ with subsequential limit $\reals \ni a^* \neq \pm 1/2$. Then for any $k$ we have that $n_k$ is either odd or even ($n_k = 2q+1$ or $n_k = 2q$ for some $q \in \naturals$), which means that for $k \to \infty$ (i.e. $q\to\infty$) we have that $a_{n_k}$ oscillates between the values $\pm 1/2$. Hence, no such $a^*$ exists.
        	
        	We find that $\pm 1/2$ are the only two accumulation points of $a_n$.
        	
        	\part We look at the sequence $b_n$ for subsequences $b_{2n}$ and $b_{2n+1}$. We find
        	\begin{align*}
        		a_{2n} &= \left(-1-\frac{1}{2n}\right)^{2n} \\
	        		&= (-1)^{2n}\left(1+\frac{1}{2n}\right)^{2n} \\
	        		&= \left(1+\frac{1}{2n}\right)^{2n} \xrightarrow{n\to\infty} e\,, \\
	        		\\
	        	a_{2n+1} &= \left(-1-\frac{1}{2n+1}\right)^{2n+1} \\
		        	&= (-1)^{2n+1}\left(1+\frac{1}{2n+1}\right)^{2n+1} \\
		        	&= -\left(1+\frac{1}{2n+1}\right)^{2n+1} \xrightarrow{n\to\infty} -e\,,
        	\end{align*}
        	
        	Using the same reasoning as described in \ref{4:A} we find that the only two accumulation points of $b_n$ are $\pm e$.
        \end{parts}
        
        \question
        For a sequence $\{a_n \} $, define $V(\{a_n \})$ the set of accumulation points. Give examples of sequences $\{a_n \}$ such that:
        
        \begin{parts}
        	\part $V(\{a_n \})$ has exactly 4 elements 
        	\part $V(\{a_n \})$ has infinitely many elements
        	\part $\star$ $V(\{a_n \}) = [0, 1]$
        \end{parts}
     \end{questions}
\end{document}