% !TeX spellcheck = en_US
\documentclass[week=6]{homework}
\usepackage{scrextend}
\date{\today}


\begin{document}
    \maketitle
    \thispagestyle{empty}
    \newpage
    \begin{questions}
		\let\firstquestion\question
		\renewcommand*{\question}{\vspace{7mm}\firstquestion}
        \firstquestion
        Let $f \colon \reals \backslash \{0\}$, $g,h \colon \reals \to \reals$ be given by
        \[
	        f(x) = 1/x\,, \qquad g(x) =
	        \begin{cases}
	        x & x \in \rationals \\
	        0 & x \not\in \rationals
	        \end{cases}\,, \qquad
	        h(x) = 
	        \begin{cases}
	        x^2 & x \in \rationals \\
	        0 & x \in \rationals
	        \end{cases}\,.
        \]
        Determine in what points the given functions are differentiable and calculate the derivative in these points.
        
        \begin{parts}
        	\part We know $f$ is differentiable in $x = a$ if and only if $\lim_{h\to 0} \frac{f(x+h)-f(x)}{h}$ exists. Let $x \in \reals \backslash \{0\}$ be arbitrary, then
        	\begin{align*}
        		\lim_{h\to 0} \frac{f(x+h)-f(x)}{h} &= \lim_{h\to 0} \frac{1/(x+h)-1/x}{h} \\
        		&= \lim_{h\to 0} \frac{1}{h}\left( \frac{-h/x}{x+h} \right) \\
        		&= \lim_{h\to 0} -\frac{1}{x(x+h)} = -\frac{1}{x^2}.
        	\end{align*}
        	
        	This implies the derivative exists for all $x \in \reals \backslash \{0\}$.
        	
        	\part \label{5:b} We know
        	\begin{align*}
        		f \text{ differentiable in } a &\implies f \text{ continuous in } a\,, \\
        		\therefore f \text{ not continuous in } a &\implies f \text{ not differentiable in } a\,.
        	\end{align*}
        	
        	If $f$ is not continuous in $a$ then
        	\[
	        	\forall_{\text{sequences } x_n \text{ in } \reals}: x_n \xrightarrow{n\to\infty} a \implies \neg\left[ f(x_n) \xrightarrow{n\to\infty} f(a)\right]
        	\]
        	Let $0 \neq a \in \reals$ be arbitrary. We distinguish two cases
        	\begin{itemize}
        		\item $a \in \rationals$
        		
        		Let $x_n$ be a sequence in $\reals \backslash \rationals$ (i.e. irrational numbers) with $x_n \xrightarrow{n\to\infty} a$. This is possible because for every proper interval in the real number there exists at least one rational and irrational number.
        		
        		Now $f(x_n) = 0$ for all $n \in \naturals$ and $f(x_n) \xrightarrow{n\to\infty} 0 \neq a = f(a)$. This means $f$ is not continuous in $a$.
        		
        		\item $a \in \reals \backslash \rationals$
        		
        		Let $x_n$ be a sequence in $\rationals$ (i.e. rational numbers) with $x_n \xrightarrow{n\to\infty} a$. This is possible because for every proper interval in the real number there exists at least one rational and irrational number.
        		
        		Now $f(x_n) = x_n$ for all $n \in \naturals$ and $f(x_n) \xrightarrow{n\to\infty} a \neq 0 = f(a)$. This means $f$ is not continuous in $a$. 
        	\end{itemize}
	        
	        However, for $x = 0$ we see that every sequence in $\reals$ converges to $f(0) = 0$, which means that $f$ is continuous in $x = 0$ and possibly also differentiable.
	        
	        However, we have
	        \[
		        \lim_{h\to 0} \frac{f(0+h) - f(0)}{h} = \lim_{h\to 0} \frac{f(h)}{h}\,,
	        \]
	        which does not exists, as $f(h)$ can be either $h$ or $0$, which is to say the limit is either $1$ or $0$.
	        
	        \part
	        We can use the same argument as in \ref{5:b} and create sequences to show that $f$ is not continuous in $a \neq 0$, only now if $a \in \rationals$ then $f(a) = a^2 \neq 0$ and if $a \in \rationals$ then $f(a) = 0 \neq a^2$.
	        
	        For $a = 0$, we have
	        \begin{align*}
	        	\lim_{h\to 0} \frac{f(0+h)-f(0)}{h} = \lim_{h\to 0} \frac{f(h)}{h}\,.
	        \end{align*}
	        Now either $f(h) = 0$, such that the limit is $0$ or $f(h) = h^2$, such that we find $\lim_{h\to 0} h = 0$. Therefore, $f$ is differentiable in $0$.
        \end{parts}
        
        \question
        Let $I \subset \reals$ be an interval. Let $f,g \colon I \to \reals$ be two $n$-times differentiable functions.
        \begin{toprove}
        	The pointwise defined product $f\cdot g$ is $n$-times differentiable and
        	\[
	        	(fg)^{(n)} = \sum_{k=0}^{n} {n\choose k} f^{(k)}g^{(n-k)}\,.
        	\]
        \end{toprove}
	    \begin{proof}
	    	We prove by induction.
	    	
	    	\textbf{Induction basis:}
	    	For $n=1$ we have
	    	\begin{align*}
	    		\lim_{x\to a} \frac{(fg)(x)-(fg)(a)}{x-a} &= \lim_{x\to a} \frac{f(x)g(x)-f(a)g(x)+f(a)g(x)-f(a)g(a)}{x-a} \\
	    		&= \left[ \lim_{x\to a}g(x) \right] \left[ \lim_{x\to a} \frac{f(x)-f(a)}{x-a} \right] + f(a) \left[ \lim_{x\to a} \frac{g(x)-g(a)}{x-a} \right] \\
	    		&= g(a)f'(a)+f(a)g'(a)\,,
	    	\end{align*}
	    	known as the product rule.
	    	
	    	\textbf{Induction hypothesis:}
	    	Assume true for some $n=q$, hence
	    	\[
		    	(fg)^{(q)} = \sum_{k=0}^{q} {q\choose k} f^{(k)}g^{(q-k)}\,.
	    	\]
	    	
	    	\textbf{Induction step:}
	    	We show the equality holds for $n=q+1$. We have
	    	\begin{align*}
		    	(fg)^{(q+1)}(x) &= \diff{}\left[ (fg)^{(q)}(x) \right] \\
		    	&= \diff{} \left[ \sum_{k=0}^{q} {q\choose k} f^{(k)}g^{(q-k)} \right] \qquad \mbox{(using the IH)} \\
		    	&= \sum_{k=0}^{q} {q\choose k} \diff{} \left[ f^{(k)}g^{(q-k)} \right] \\
		    	&= \sum_{k=0}^{q} {q\choose k} \left[f^{(k)}g^{(q-k+1)} + f^{(k+1)}g^{(q-k)}\right] \\
		    	&= \sum_{k=0}^{q} {q\choose k} f^{(k)}g^{(q-k+1)} + \sum_{k=0}^{q} {q\choose k} f^{(k+1)}g^{(q-k)} \\
		    	&= \sum_{k=0}^{q} {q\choose k} f^{(k)}g^{(q-k+1)} + \sum_{k=1}^{q+1} {q\choose k-1} f^{(k)}g^{(q-k+1)} \\
		    	&= fg^{(q+1)} + f^{(q+1)}g + \sum_{k=1}^{q} {q\choose k} f^{(k)}g^{(q-k+1)} + \sum_{k=1}^{q} {q\choose k-1} f^{(k)}g^{(q-k+1)} \\
		    	&= fg^{(q+1)} + f^{(q+1)}g + \sum_{k=1}^{q} {q+ 1 \choose k} f^{(k)}g^{(q-k+1)} \\
		    	&= \sum_{k=0}^{q+1} {q+ 1 \choose k} f^{(k)}g^{(q-k+1)}\,,
	    	\end{align*}
	    	using some properties of the binomial coefficients, namely;
	    	\begin{gather*}
	    		{n \choose k} + {n \choose k-1} = {n+1 \choose k} \,, \\
	    		{n \choose 0} = 1 \,, \qquad {n \choose n} = 1\,.
	    	\end{gather*}
	    	This completes the proof.
	    \end{proof}
     \end{questions}
\end{document}