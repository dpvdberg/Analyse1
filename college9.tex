% !TeX spellcheck = en_US
\documentclass[week=6]{homework}
\usepackage{scrextend}
\date{\today}


\begin{document}
    \maketitle
    \thispagestyle{empty}
    \newpage
    \begin{questions}
		\let\firstquestion\question
		\renewcommand*{\question}{\vspace{7mm}\firstquestion}
        \firstquestion
        a
        \question
        Let $I \subset \reals$ be an interval. Let $f,g \colon I \to \reals$ be two $n$-times differentiable functions.
        \begin{toprove}
        	The pointwise defined product $f\cdot g$ is $n$-times differentiable and
        	\[
	        	(fg)^{(n)} = \sum_{k=0}^{n} {n\choose k} f^{(k)}g^{(n-k)}\,.
        	\]
        \end{toprove}
	    \begin{proof}
	    	We prove by induction.
	    	
	    	\textbf{Induction basis:}
	    	For $n=1$ we have
	    	\begin{align*}
	    		\lim_{x\to a} \frac{(fg)(x)-(fg)(a)}{x-a} &= \lim_{x\to a} \frac{f(x)g(x)-f(a)g(x)+f(a)g(x)-f(a)g(a)}{x-a} \\
	    		&= \left[ \lim_{x\to a}g(x) \right] \left[ \lim_{x\to a} \frac{f(x)-f(a)}{x-a} \right] + f(a) \left[ \lim_{x\to a} \frac{g(x)-g(a)}{x-a} \right] \\
	    		&= g(a)f'(a)+f(a)g'(a)\,,
	    	\end{align*}
	    	known as the product rule.
	    	
	    	\textbf{Induction hypothesis:}
	    	Assume true for some $n=q$, hence
	    	\[
		    	(fg)^{(q)} = \sum_{k=0}^{q} {q\choose k} f^{(k)}g^{(q-k)}\,.
	    	\]
	    	
	    	\textbf{Induction step:}
	    	We show the equality holds for $n=q+1$. We have
	    	\begin{align*}
		    	(fg)^{(q+1)}(x) &= \diff{}\left[ (fg)^{(q)}(x) \right] \\
		    	&= \diff{} \left[ \sum_{k=0}^{q} {q\choose k} f^{(k)}g^{(q-k)} \right] \qquad \mbox{(using the IH)} \\
		    	&= \sum_{k=0}^{q} {q\choose k} \diff{} \left[ f^{(k)}g^{(q-k)} \right] \\
		    	&= \sum_{k=0}^{q} {q\choose k} \left[f^{(k)}g^{(q-k+1)} + f^{(k+1)}g^{(q-k)}\right] \\
		    	&= \sum_{k=0}^{q} {q\choose k} f^{(k)}g^{(q-k+1)} + \sum_{k=0}^{q} {q\choose k} f^{(k+1)}g^{(q-k)} \\
		    	&= \sum_{k=0}^{q} {q\choose k} f^{(k)}g^{(q-k+1)} + \sum_{k=1}^{q+1} {q\choose k-1} f^{(k)}g^{(q-k+1)} \\
		    	&= fg^{(q+1)} + f^{(q+1)}g + \sum_{k=1}^{q} {q\choose k} f^{(k)}g^{(q-k+1)} + \sum_{k=1}^{q} {q\choose k-1} f^{(k)}g^{(q-k+1)} \\
		    	&= fg^{(q+1)} + f^{(q+1)}g + \sum_{k=1}^{q} {q+ 1 \choose k} f^{(k)}g^{(q-k+1)} \\
		    	&= \sum_{k=0}^{q+1} {q+ 1 \choose k} f^{(k)}g^{(q-k+1)}\,,
	    	\end{align*}
	    	using some properties of the binomial coefficients, namely;
	    	\begin{gather*}
	    		{n \choose k} + {n \choose k-1} = {n+1 \choose k} \,, \\
	    		{n \choose 0} = 1 \,, \qquad {n \choose n} = 1\,.
	    	\end{gather*}
	    	This completes the proof.
	    \end{proof}
     \end{questions}
\end{document}